 \stepcounter{lecture}
 \setcounter{lecture}{18}
 \sektion{Lecture 18}

 \begin{theorem}[Schur-Horn]
   If you look at the action of $U(n)$ on $n\times n$ hermitian matrices, and let
   $\O_{(\lambda_1,\dots, \lambda_n)}$.

  $conv\{(/lambda_{\sigma(1)},\dots)\}\{diag(ADA^*)\}\in \RR^n$, where
  $D=diag(\lambda_1,\dots, \lambda_n)$ and $\sigma$ a permutation.
 \end{theorem}

 Then Kostant generalized, Heckman reinterpreted in terms of coadjoint orbits,
 Guilleman-Sternberg reinterpreted as tori acting on symplectic manifolds, Atiyah some
 other stuff.  Then Kirwan generalized for non-tori ... N.T.Zung ``the ultimate
 convexity theorem''.

 Ingredients of proof
 \begin{itemize}
 \item[(1)] Local normal form $\Rightarrow$ local convexity
 \item[(2)] ``Morse theory'' $\rightarrow$ global result
 \item[(3)] Covering argument (C-D-M) (connected fiber
 \end{itemize}

 $x$ fixed point for the torus action, action on $T_xM$ linear, $G$-equivariant as in
 last lecture.

 Then we find a $G$-invariant hermitian structure on $T_xM$.

 $G$ compact compact acting on vector space $V$.  Then there is a $G$-invariant inner
 product on $V$.  To see this, take any inner product $B$ and define
 \[
    \bar B(v,w) = \int_G B(gv,gw)\, dg
 \]
 where $dg$ is the haar measure.

 This gives us a compatible complex structure.  Now we are going to use that $G$ is a
 torus $\mathbb{T}^m$, so we can diagonalize the action by choosing coordinates
 $q_j,p_j$.  We get a map $\t^m\to \RR^n\simeq \t^n$.  Think of $G=\RR^m/2\pi \ZZ^m$,
 then this map is an integer matrix.

 $\langle J(z^1,\dots, z^n),(t_1,\dots, t_n)\rangle = \frac{1}{2}t_jr^j_k|z^k|^2 =
 \frac{1}{2}t_jr^j_k((q^k)^2+(p^k)^2)$, where $r^j_k$ is an integer matrix.
 Hamilton's equations:
 \begin{align*}
   \dot{q}^k &= \pder{\J}{p^k} = t_jr^j_kp^k & (\text{not sum over $k$})\\
   \dot{p}^k &= -\pder{\J}{q^k} = -t_jr^j_kq^k &(\text{not sum over $k$})
 \end{align*}
 .  So $J(z^1,\dots, z^n)=\{r^j_k\cdot
 \frac{1}{2}|z^k|^2\}=r^j_k\frac{1}{2}|z^k|^2\tau_j =
 \underbrace{(r^j_k\tau_j)}_{e_k}\frac{1}{2}|z^k|^2$, where $\tau_j$ is the $j$-th
 basis vector of $\t^*$, $e_k$ a vector in $\t^*$ with integer components.  Thus,
 locally, the image of the momentum map are all the non-negative linear combinations
 of the $e_k$'s.  This gives us a polyhedral cone (if the $e_k$ are pointing in
 sufficiently different directions that our cone is everything).

 How do you know there are any fixed points?
 \begin{lemma}
   For any torus action (in fact, any compact lie group), the fixed point set is a
   symplectic submanifold.
 \end{lemma}
 This follows from our normal form near a fixed point (the fixed set must be a complex
 subspace, hence a symplectic submanifold).  Then $\mathbb{T}^m=\mathbb{T}^1\times
 \cdots$. The first has a fixed point because the action is hamiltonian, and the
 hamiltonian function must have a max and min ... then use induction.

 We have that the image of $J$ is locally in a bunch of these polyhedral cones.  Now
 we use morse theory to show that the image is globally in the intersection of these
 cones.

 Figure 1

 look at the flow of $\exp(tv)$ on $M$.  What is its hamiltonian?  It is given by
 composing $v$ with the momentum map.  The hamiltonian has local maximum at $x$, and
 it must have some global maximum, but our local max isn't the global max because
 $J(y)$ is on the other side of the plane, contradicting
 \begin{lemma}
   For any 1-parameter subgroup of the torus, its hamiltonian has a unique local
   maximum value.
 \end{lemma}
 \begin{proof}
   At each critical point, the function is quadratic with an even number of positive
   coefficients.  The maxima are the points where that even number is zero.  Connect
   two local maxima with a path.  Along the path, there is a minimum value ... choose
   the path with the largest minimum.  This must be a critical point, otherwise, you
   could find a better path.  There cannot be two directions in which you increase,
   otherwise you could slide around the bowl and get a better path.  Thus, there must
   be a point where there is only one positive direction (or none).  This proves that
   the set of local maxima is connected.
 \end{proof}

 This tells us that the image is in the intersection of all the polyhedral cones.
 Suppose there were some inside point not in the image, then we may take the largest
 ball which is in the complement of the image.  This touches the
 image somewhere.  Look at the tangent hyperplane at the point of contact.  We will
 show that the whole image is on the other side of the plane, which contradicts the
 assumption that we took the largest ball.

 $x\in M$.  $\mathbb{T}_x=$ stabilizer of $x$ ... this is a closed subgroup of the
 torus. $M^{\TT_x}\subseteq M$ the set of all points fixed by $\TT_x$.

 figure 2

 The composite map is the momentum map for the action of $\TT_x$.  The image in
 $\t_x^*$ is in a cone, so the image of $J$ is in the ``wedge'' in $\t^*$.  Now we
 need to show that $J$ fills up the wedge around $J(x)$.  $M^{\TT_x}\subseteq M$
 symplectic submanifold.  Lets look at $J(M^{\TT_x})$ ... it must lie in the pre-image
 of the image of $J(x)$.  It fills up that pre-image because we can find a
 complementary torus to $\TT_x$, which must be free around $x$, so blah.  Now the same
 morse theory argument shows that the image is globally in the wedge.

 End of proof of convexity of image of $J$.

 Now we need to show connectivity of the fibers. Define an equivalence relation on
 $M$.  Equivalence classes are connected components of fibers of $J$. Let $\tilde
 M=M/\sim$.
 \[\xymatrix{
 \tilde M  \ar[dr]^{\tilde J}\\
 M\ar[u]\ar[r]^J & J(M) \ar@{}[r]|{\subseteq} & \g^*
 }\]
 and we want to show that $\tilde J$ is bijective.  Notice that $\tilde M$ is
 connected because it is a quotient of a connected set.  $J(M)$ is simply connected
 because it is convex. so it is enough show that $\tilde J$ is a covering map.

 1)$\tilde J$ is a local homeomorphism (from normal form).\\
 2) if $f:X\to Y$ is local homeomorphism and $f$ proper, then $g$ is a covering map.
To see this,

 figure 3

 since the map is proper, the inverse image of a point is finite.  each point in the
 inverse image has a nbd which maps homeomorphically to a nbd of $y$ ... intersect
 those.  We just need that we have the complete inverse image of our nbd of $y$.
 Suppose not.  Then there is a sequence $y_i\to y$ and $x_i\in f^{-1}(y_i)$ such that
 $x_i\not\in $any $U_j$.  $\{y_i,y\}$ compact, so by properness, the $x_i$ have a
 convergent subseq, which must be in the inverse image of $y$.  We win!
