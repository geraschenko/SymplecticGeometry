 \stepcounter{lecture}
 \setcounter{lecture}{17}
 \sektion{Lecture 17}

 Did everybody get the email saying that you shouldn't use the bspace dropbox? Good.

 We start with $(G,\w,G,J)$, $J:M\to \g^*$ with corresponding $\J:\g\to C^\infty(M)$.
 Now we define
 \[
    \theta(v,w):=\{\J(v),\J(w)\}-\J([v,w]) \in \RR \quad (\text{generally, } H^0(M)).
 \]
 This is a bilinear, skew-symmetric map, $\theta: \g\wedge\g\to H^0(M)$.  It satisfies
 a kind of Jacobi identity:
 \[
    \delta\theta(v,w,u)=\theta([v,w],u)+(\text{cyclic permutations})=0.
 \]
 This is a piece of a complex which calculates the lie algebra cohomology.

 $C^k(\g)=\wedge^k\g^*=\hom(\wedge^k \g,\RR)$, $\delta: C^k(\g)\to C^{k+1}(\g)$ given
 by
 \begin{align*}
 (\delta\theta)(v_0,\dots,v_k)= - \sum_{i<j}(-1)^{s(i,j)}\theta([v_i,v_j],v_0,\dots, \hat
 v_i,\dots, \hat v_j,\dots ,v_k).
 \end{align*}
 This has a lot to do with the de Rham
 differential.
 [[\begin{align*}
 (\delta\theta)(v_0,\dots,v_k)= - \sum_{i<j}&(-1)^{s(i,j)}\theta([v_i,v_j],v_0,\dots, \hat
 v_i,\dots, \hat v_j,\dots ,v_k)\\
 &+v_i\cdot \theta(v_0,\dots, \hat v_i,\dots, v_k).
 \end{align*}]]

 If $G$ is a compact, connected lie group, then $H^k_{dR}(G,\RR)\simeq H^k(\g,\RR)$.  If
 $G=\mathbb{T}^n$, then $\g=\RR^n$, and $C^k(\g)=\wedge^k(\RR^n)^*=H^k(\g)$, which has
 dimension $\binom{n}{k}$.
 If $G=\RR^n$, then it has the same lie algebra, but the cohomology is different.  So
 you definitely need compact and connected.

 Recall that we can change $\J$ by locally constant functions.  Say we replace
 $\J\rightsquigarrow \J+b$, where $b:\g\to H^0(M,\RR)$ (i.e. $b\in \g^*\otimes
 H^0(M,\RR)$).  Then
 \begin{align*}
  \theta_{\J+b}(v,w) &= \{\J(v)+b(w),\J(w)+b(v)\}-\J([v,w])-b([v,w])\\
  &= \theta_\J (v,w) + \underbrace{\{\J(v),b(w)\} + \{b(v),\J(w)\}}_{0+0}- \underbrace{b([v,w])}_{\pm \delta b(v,w)}.
 \end{align*}
 So when we change $\J$ by a locally constant function, the $\theta$ is changed by a
 coboundary.  Thus, given an action admitting a comomentum map, there is an associated
 element of $H^2(\g,H^0(M))$ which vanishes if and only if there is a comomentum map
 which is a lie algebra homomorphism.

 Last time we showed
 \begin{align*}
   \J(Ad_g v)-{g^{-1}_M}^*\J(v) &= \langle \overbrace{c(g)}^{\in \g^*},v\rangle
 \end{align*}
 for some $c:G\to \g^*$.  Now we're going to differentiate $c$ at the origin.  Let
$g=\exp(tw)$ for $w\in \g$, then take $\der{}{t}$ at $t=0$.  Then we get
\begin{align*}
  \J([w,v])+X_{\J(w)}\cdot \J(v) &= \J([w,v])+\{\J(v),\J(w)\} \\
  &= \theta(v,w)
\end{align*}
on the LHS, and on the RHS, we have $\langle (T_e c)(w),v\rangle$.  So if we think of
$c$ as a bilinear form on $\g$, it is exactly $\theta$.  It turns out that if
$\theta=0$, you get that $c=0$ on the connected component of the identity.  So on a
connected group, $\theta$ and $c$ contain the same information.  Not so on a
non-connected group:

Consider $O(3)$ acting on $T^*\RR^3$ with coordinates $q^i,p_i$ $i=1,2,3$.  Take the
one parameter group of rotations given by rotation of the $1$-$2$ plane.
$q^1\pder{}{p_1} -q^2\pder{}{p_2}$, with hamiltonian $q^1p_2-q^2p_1$.  Then $J=q\times
p$. $L(q)\times L(p)=L(q\times p)$.  The momentum map is $SO_3$ equivariant, but not
$O_3$ equivariant because reflection introduces a minus sign.  We've identified
$\mathfrak{o}_3\cong \mathfrak{so}_3\simeq \RR^3$.

 Another example: let $G$ be a lie group, then $G$ acts on $G$ by right translations.
 That is, $g\rightsquigarrow r_{g^{-1}}$.  Then we can lift to $T^*G$: $g\cdot \alpha =
 r_{g}^*\alpha$.  Since any lifted action is symplectic, we can expect a momentum map,
 and we can expect it to be linear on fibres.  If you look at a one-parameter
 subgroup, the infinitesimal generator is a left-invariant vector field.  $J:T^*G\to
 \g^*$ given by left translation back to the identity.  Then $J^{-1}(\mu)$ is all
 cotangent vectors, which when translated back to the identity become $\mu$, so this
 is just the image of the left invariant 1-form $\mu$, which is isomorphic to $G$ as a
 manifold.  So what is $J^{-1}(\mu)/G_\mu$?  It is $G/G_\mu$, which is the coadjoint
 orbit of $\mu$.  Thus, $G/G_\mu=G\cdot \mu$ inherits a reduced symplectic structure.

 \[\xymatrix{
 T^*G \ar@{<-^)}[r] \ar[d] & J^{-1}(\mu) \ar[d]\\
 T^*G/G \ar@{<-^)}[r] & J^{-1}(\mu)/G_\mu
 }\]
 Here, $T^*G/G$ is a poisson manifold, with the linear lie-poisson structure on
 $\g^*$: $\{\mu_i,\mu_j\}=c_{ij}^k\mu_k$.

 \subsection*{Convexity Theorem}
 Two big theorems in symplectic geometry, the Atiyah-Guilleman/Sternberg convexity
 theorem (1982), and Gromov pseudoholomorphic curve theorem (1985).  We will talk
 about the first theorem

 \begin{theorem}
   $(M,\w)$ connected symplectic, $G=\mathbb{T}^m=(\RR/\ZZ)^m$ acting on $M$ with equivariant
   $J:M\to \g^*\cong \RR^m$.  [[If $v_1,\dots, v_m$ a basis for the lie algebra
   $\mathfrak{t}^m$ of $\mathbb{T}^m$, then $\{\J(v_1),\J(v_2)\}=0$.  Then the orbits
   of $v_{i,M}$ are isotropic.]]  Then the $J$ level submanifolds are connected, and
   $J(M)\subseteq \mathfrak{t}^{m*}$ is the convex hull of $J(M^G)$ (which is a finite
   set).
 \end{theorem}
 If $m=1$, we have a circle action, $J:M\to \RR$ is just the hamiltonian for the
 circle action.  Then the theorem says that the image is a closed interval whose
 endpoints are fixed points.

 \underline{Local version}: Assume we have a fixed point $x\in M$.  Locally, we have a
 symplectic torus action on $\RR^{2n}$ containing $0$ as a fixed point.
 \begin{theorem}[Bachner]
   Any smooth action of a compact group may be linearized around any fixed point.
   i.e., given $x\in M$ fixed by a compact group action, there is some coordinate
   system around $x$ such that the action on the coordinate system is linear.
 \end{theorem}
 Choose a Riemannian metric on $M$, then use compactness of the group to replace the
 metric by an invariant metric under the action of the group.  Then use the
 exponential map to give a coordinate chart.  Then the action sends geodesics to
 geodesics, so on the tangent space, it is linear.

 So we can say that around a fixed point, the torus action is linear.  We would like
 to say that it is symplectic, but it is only symplectic with respect to $\exp^*\w$,
 where $\w$ is the symplectic form on $M$.  We would like: if we choose something
 other than $\exp$, $\w$ will pull back to a constant symplectic form.  What we really
 need is a Bochner-Darboux theorem, which says that if we have a compact group acting
 on a symplectic manifold near a fixed point, we can find a coordinate system where
 the group action is linear and the symplectic form is constant.  That is not too hard
 to prove using the Moser method.  First use the Bochner theorem.  Then we have two
 symplectic structures on our $T_xM$.  Then we need to take one to the other in such a
 way that it commutes with the action of the group.  But if you look at the proof of
 the Darboux theorem, the construction commutes with any linear action.  This gives us
 a linear symplectic action near a fixed point.  Now we need to look at the momentum
 map.
