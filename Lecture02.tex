 \stepcounter{lecture}
 \setcounter{lecture}{2}
 \sektion{Lecture 2}

\noindent Office hours:

Alan W.: 825 Evans, Tu 12:40 - 2, Th 9:40 - 11

Chistian B.: 898 Evans, Tu 9:40 - 11

Questions from last time or from the reading:

 \begin{itemize}
 \item[(a)] If we have $V_1,V_2$, then isomorphisms between $V_1$
 and $V_2$ correspond to lagrangians in $\bar V_1\oplus V_2$.
 Lagrangians in $E\oplus E^*$ (with $\W_{+[-]}$) correspond to
 skew-symmetric forms [symmetric forms] on $E$.

 How do we put these together?  If $W = \bar V_1\oplus V_2\cong
 E\oplus E^*$.  Then there are two big subsets of $\lag(W)$, one
 corresponding to isomorphisms $V_1\to V_2$, and the other
 corresponding to forms on $E$.

 \[
  \begin{pspicture}(0,-1)(9,1.8)
  \psccurve(0,0)(4.5,-1)(9,0)(9,1)(4.5,2)(0,1)
  \psccurve(.5,.5)(3,-.5)(5.5,.5)(3,1.5)
  \psccurve(3.5,.5)(6,-.5)(8.5,.5)(6,1.5)
  \rput(9,1.8){$\lag(W)$}
  \rput(1.8,.5){iso$(V_1,V_2)$}
  \rput(7,.5){${{}_{\substack{\text{[skew-]symm} \\ \text{forms on $E$}}}}$}
  \end{pspicture}
 \]

 If $V$ has a (non-degenerate) ``pure''\footnote{Either
 skew-symmetric or symmetric.} bilinear form $B$, then we can talk
 about $\lag(V)\subseteq Gr_{1/2}(V) = $ half-dimensional
 subspaces, and it is always a submanifold.  To see that
 $Gr_{1/2}(V)$ is a manifold, note that it is homogeneous under
 the action of $GL(V)$.  That is, we have a map
 $GL(V)\twoheadrightarrow Gr_{1/2}(V)$, where the kernel is stuff
 like $\matrix{*}{*}{0}{*}$.

 \begin{theorem}[Witt's Theorem]
  $\aut(V,B)$ acts transitively on $\lag(V)$.
 \end{theorem}

 So we can show that $\lag(V)$ is a manifold.  If we take the case
 $V=E\oplus E^*$, the elements of $\lag(V)$ not intersecting $E^*$
 are bilinear forms (skew or symmetric, depending on $\W_{\pm}$).
 Thus, we have a bijection between such forms and an open subset
 of $\lag(V)$.

 Thus, in the skew case, $\lag(V)$ has dimension
 $\frac{n(n+1)}{2}$, and in the symmetric case, it is
 $\frac{n(n-1)}{2}$.

 \item[(b)]  Suppose we have $V,W,X$ all pure of the same type.
 And say $L_1\subseteq \bar V\times W$ and $L_2\subseteq \bar
 W\times X$ lagrangians.  Then we can form $L_2\circ L_1\subseteq
 \bar V\times X$.  It is the set of all $(v,x)\in \bar V\times X$
 such that there is a $w\in W$ such that $(v,w)\in L_1$ and
 $(w,x)\in L_2$.  This kind of composition is defined for
 arbitrary relations.  It is not hard to see that if $L_1$ and
 $L_2$ are linear, then so is $L_2\circ L_1$.  What is amazing is
 that if if $L_1$ and $L_2$ are lagrangian, then so is $L_2\circ
 L_1$.$^*$\marginpar{$^*$Optional Problem: prove this}

 Furthermore, in $\bar V\times V$, the diagonal, $\Delta_V$, is
 lagrangian (clear), and it is the identity under composition
 (when defined\footnote{That is, when the domains and ranges match
 up.}).

 Thus, we have a category.  The objects are all pure,
 non-degenerate vector spaces, and $\hom(V,W) = \lag(\bar V\times
 W)$ (and it is the empty set if $V$ and $W$ have different
 types).  Two things are isomorphic in this category if and only
 if they are isomorphic in the usual sense.  We have added some
 homomorphisms.

 Example: What are the homomorphisms between $0$ and $V$?  They
 are just lagrangians in $\bar 0 \times V = V$, and
 $\hom(V,0)=\lag(\bar V)$.

 When is $\hom(V,W)$ non-empty?  Well, $V$ and $W$ have to have
 the same type.  Then the question is, ``when does $\bar V\times
 W$ have lagrangians?''
 \begin{lemma}
   In the skew case, $\lag(\text{anything})\not= \varnothing$.\\
   In the symmetric case, if over $\mathbb{C}$, same as skew
   symmetric.  If we are over $\mathbb{R}$, we have lagrangians
   only if the signiture\footnote{Number of positive eigenvalues
   minus the number of negative ones.} is zero.
 \end{lemma}
 You can take a lagrangian in $\bar V$ and a lagrangian in $W$,
 you can cross them.  When you take direct products, signatures
 add, so in the real symmetric case, you need to have
 $sign(V)=sign(W)$.
 \end{itemize}

 \section*{Differentiable Manifolds}

 All this linear algebra should be thought of as taking place in
 the tangent spaces of manifolds.  All our manifolds are
 $C^{\infty}$.

 \underline{2-forms:}  $\w\in \W^2(M) = \Gamma(\wedge^2T^*M)$
 is the set of 2-forms on $M$.  We can make $\tilde \w:TM\to
 T^*M$, given by $\tilde \w(v)(w) = \w(v,w)$, which will be a
 smooth map of bundles.  This is also sometimes written $\tilde
 \w(v) = i_v\w = v \lrcorner \w$.  In fact,
 $i_v:\wedge^pT^*_xM\to \wedge^{p-1}T^*_xM$ for $v\in T_xM$
 by putting $v$ in for the first entry.

 In local coordinates $(x^1,\dots, x^m)$, we can write
 \[
    \w = \frac{1}{2}\w_{ij}(x)dx^i\wedge dx^j
 \]
 where $\w_{ij}=-\w_{ji} = \w\left(\frac{\partial}{\partial
x^i},\frac{\partial}{\partial x^j}\right)$.

 $\w\in \W^2(M)$ is \emph{non-degenerate} if $\tilde \w$ is
 invertible, and we say it is \emph{presymplectic} if $d\w=0$
 (i.e. $\w$ closed).  We say it is \emph{symplectic} if it is
 both.

 \underline{Example:} On $\mathbb{R}^{2n}$, with coordinates
 $(q^1,\dots, q^n,p_1,\dots, p_n)$, take $\w_n = dq^i\wedge dp_i := \sum_i dq^i\wedge dp_i$.
 It is clear that $\w$ is presymplectic.  In matrix form, it is
 $\matrix{0}{I}{-I}{0}$.

 \begin{lemma}
   $\w$ is non-degenerate if and only if $\w^{\dim M/2} = \w\wedge\cdots\wedge \w$ is
   nowhere zero in $\W^{\text{top}}(M)$.
 \end{lemma}
 In our case, we have $(\w_n)^n = n!\, dq^1\wedge dp_1\wedge\cdots\wedge dq^n\wedge
 dp_n$, which is almost the canonical volume form (which is
 $dq^1\wedge\cdots \wedge dq^n\wedge dp_1\wedge\cdots \wedge
 dp_n$).

 The \emph{symplectic volume} is either $\frac{1}{n!}\w^n$ or
 $(-1)^{\text{something}} \frac{1}{n!}\w^n$, according to
 convention ... generally the first one ($2n=\dim M$).  Not every
 manifold can support a symplectic structure ... it has to be even
 dimensional, and orientable!  Look at $[\w]\in
 H^2_{DR}(M,\mathbb{R})$.  We have that $[\w]^n=[\w^n]$.  If $M$
 is compact, then
 \[
    \int_M\w^n > 0
 \]
 so then $\w$ is not exact.  Thus, $M$ must have some cohomology
 in degree 2 too.  You also have to have a non-degenerate
 2-form, which is equivalent to an \emph{almost complex}
 structure (i.e., a map $J:TM\to TM$ such that $J^2=-I$).

 It took a while to find an example satisfying these conditions,
 but not having a symplectic structure.  Take the connected sum
 $\CC P^2 \#\CC P^2\#\CC P^2$.  This manifold has lots of cohomology
 in degree 2 with the right squares, and it has an almost complex
 structure, but it doesn't have a symplectic structure (some
 invariant is non-zero ... Sieberg-Witten stuff).  There is a
 theorem that any symplectic manifold locally looks like the
 example we gave above (with $\w_n$ on $\mathbb{R}^{2n}$).  Note
 that
 \[
    d\w = \frac{1}{2}\frac{\partial \w_{ij}}{\partial x^k}
    dx^k\wedge dx^i\wedge dx^j.
 \]
 The vanishing of this expression doesn't ensure the vanishing of
 $\frac{\partial \w_{ij}}{\partial x^k}$.  It turns out that it is zero if and
 only if
 \[
 \frac{\partial \w_{ij}}{\partial x^k} +
 \frac{\partial \w_{jk}}{\partial x^i} +
 \frac{\partial \w_{ki}}{\partial x^j} = 0
 \]

 There is a slightly more general version which says that $d\w =0$
 and $\tilde \w$ has constant rank (i.e. if the matrix
 $\w_{ij}(x)$ has constant rank), then in suitable coordinates
 $(q^1,\dots, q^r,p_1,\dots, p_r,\lambda^1,\dots, \lambda^s)$, we
 have
 \[
    \w = dq^i\wedge dp_i.
 \]
 In these coordinates, the kernel of $\tilde \w$ is the span of
 the $\frac{\partial} {\partial \lambda^i}$.  In general, we have
 $\ker \tilde \w \subseteq TM$.  If $\w$ has constant rank, these
 spaces all have the same dimension, and they define a smooth
 sub-bundle of $TM$.  In this case, one can prove that
 \[
    d\w=0 \Rightarrow \ker \tilde \w \text{ is involutive}
 \]
 which implies (by the Frobenius theorem) that $\ker\tilde \w$ is
 tangent to a ``characteristic'' foliation of $M$.  In this
 picture, the $\lambda^i$s are the coordinates (locally) along the
 foliation, and the $p_i$s and $q^i$s are transverse to the
 foliation.  We can get back to the symplectic case by throwing
 away the $\lambda^i$s.
