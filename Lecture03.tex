 \stepcounter{lecture}
 \setcounter{lecture}{3}
 \sektion{Lecture 3}

 Questions from last time:
 \begin{itemize}
 \item[-] Foliations
 \item[-] de Rham cohomology
 \item[-] Lagragian subspaces as morphisms. If $V$ and $W$ are
 of the same type, and the signatures are the same, then there is
 an injection $Iso(V,W)\hookrightarrow \lag(\bar V\times W) = \hom(V,W)$.  If
 $V=W$, then $\aut V\hookrightarrow \lag(\bar V\times V)$.  This
 is a semigroup with identity, which contains $\aut V$ as a
 subgroup.  In fact, these are just the invertible elements.  What
 can we say about this semigroup? e.g., say $V=\RR^3$ Euclidean,
 which gives $O(3)$ as automorphisms, which is an open subset of
 $\lag(\bar V\times V)$, and it is closed since it is compact.
 I think $O(3)$ is all of $\lag(\bar V\times V)$.  To show this,
 we need to show that no lagrangian subspace intersects the $V$
 axis.\marginpar{really?}

 If $V=\RR^2$ symplectic, then the automorphisms are $SL_2(\RR)$ (non-compact),
 and $\lag(\RR^4)$ look like $2\times 2$ symmetric matrices.  The
 automorphism group sits in $lag(\RR^4)$ as an open dense subset.

 \underline{Warning:} The product on $\lag(\bar V\times V)$ is
 not, in general continuous!

 \marginpar{Optional: analyze this multiplication}

 C. Sabot: somehow related lagrange grassmanian to probability,
 and modified this so that the multiplication is continuous.

 \item[-] The non-symplectic example $(\CC P^2)^{\# 3}$ ... more?
 I dunno.
 \end{itemize}

 Important things from the reading:

 On the cotangent bundle $T^*X$ of any manifold, there is
 a natural structure ... coordinates $(x^i,\xi_i)$.  $\alpha=\xi_i
 dx^i$ and $\w=-d\alpha = dx^i\wedge d\xi_i$.  If $\phi$ is any
 1-form (section of cotangent bundle), $\phi^*\alpha=\phi$, and
 this characterizes $\alpha$.  This $\alpha$ is called the
 \emph{canonical 1-form} or the \emph{Liouville form}.  You can
 see from local coordinates that $\w$ gives a symplectic
 structure.

 $T_{(x,0)}T^*X \cong T_xX\oplus (T_xX)^*$, which will always have
 a natural symplectic structure $\W_-$.  This works along the zero
 section, but elsewhere, you don't have a canonical
 identification with a space and its dual, so you have to go
 through the canonical 1-form.  There are some nice ways to
 characterize the canonical 1-form, but it is not so easy to see
 the 2-form.

 In Chapter 3, we talk about symplectic maps, and
 lagrangian submanifolds of symplectic manifolds.  Let $(M,\w)$ be
 symplectic. Then say $N \stackrel{i}{\looparrowright} M$ is \emph{an
 immersion} if $Ti$ injects into $T_{pt}M$.  We say $i$ is
 [co]isotropic [lagrangian] if $(T_xi)(T_xN)\subseteq T_{i(x)}M$
 is.  That is, $i$ isotropic if $i^*\w=0$, etc.

 What we called $\bar V\times V$ before is now $\bar M\times M$,
 where $\bar M$ is the same manifold, but with the sign of the
 symplectic structure changed.  Then $(\bar M\times
 M,-\pi^*_1\w+\pi^*_2\w)$, where the $\pi_i$ are the natural
 projections.  We can also look at $\bar M_1\times M_2$, where
 $M_1$ and $M_2$ are different.  Remember that the graphs of
 morphisms $V_1\to V_2$ were lagrangians in $\bar V_1\times V_2$.
 The same is true here.  If $f:M_1\to M_2$, then the graph of $f$
 is lagrangian in $\bar M_1\times M_2$ if and only if
 $f^*\w_2=\w_1$ and $\dim M_1=\dim M_2$, which implies that $f$ is
 an immersion (otherwise, $f^*\w_2$ would be degenerate).  All of
 this implies that $f$ is a local diffeomorphism.  We call such a
 map a local symplectomorphism, and if it is a global
 difeomorphism, then it is called a symplectomorphism.  The
 conclusion is that the graph of $f$ is lagrangian if and only if
 $f$ is a local symplectomorphism.  Now we can look at $\lag(\bar
 M_1\times M_2)$, the set of lagrangian submanifolds.  Inside of
 it are the local symplectomorphisms $Locsymp(M_1,M_2$ (in
 particular, the symplectomorphisms).  We have that $\aut (M,\w)
 \hookrightarrow \lag(\bar M_1\times M_2$.  Ok, can we compose
 lagrangian submanifolds?  Well, you don't always get a manifold
 (this relates to the non-continuity problem we talked about
 earlier).  It still makes sense to think of the elements of
 $\lag(\bar M_1\times M_2)$ as generalized morphisms, and call it
 ``$\hom$''$(M_1,M_2)$ in some category.  Note that even if the
 dimensions don't match, we can have some lagrangian submanifolds.
 In particular, if $M_1=pt$, then ``$\hom$''$(pt,M)=\lag(M)$.  So in
 this category, the role of points is played by lagrangian
 submanifolds.

 pictures

 The other interesting case in the vector space case was $E\oplus
 E^*$.  This corresponds to asking about the lagrangian
 submanifolds of $T^*X$.  Well, there are two kinds of
 submanifolds of $T^*X$. There are the fibers, along which the
 canonical 1-form vanishes, an therefore so does the 2-form.  The
 other kind of submanifold is the graph of a 1-form.  When is it
 lagrangian?  Let's call the 1-form $\phi$.
 \[
   \phi \text{ a lagrangian immersion} \Leftrightarrow
   0=\phi^*(-d\alpha) = -d(\phi^*\alpha) = -d\phi
 \]
 So $\phi$ must be closed.  Locally, we must have $\phi = dS$
 where $S\in C^{\infty}(X)$.  We can change $S$ by any locally
 constant function.  Thus, lagrangian sections of $T^*X$ are more
 or less $C^{\infty}(X)/$consts.  Of course, if $X$ is not simply
 connected, then there should be some correction (since not all
 closed 1-forms are of the form $dS$).

 Thus, we should think of $\lag(T^*X)$ as ``generalized
 functions''.  We also see these in probability ... e.g. the Dirac
 delta distribution should correspond to a fiber of the bundle.
 What about as lagrangian section?  Some references: Bates -W. -
 Lectures on the geometry of quantization.; Guillemin-Sternberg -
 Geometric asymptotics.

 Take the example $X=\RR$.  Take a lagrangian subspace of $T^*X$,
 then this has the form $\xi = ax$ for some $a$, which corresponds
 to the closed 1-form $a\,dx = d(\frac{1}{2} ax^2)$
 \marginpar{something is funny here.}.  In the case
 $f=\frac{\partial s}{\partial x}$, we have the corresponding
 something $e^{\frac{i}{2} ax^2}$.  As $x\to \infty$, this
 oscillates faster and faster, and as $a\to \infty$, the smooth
 part gets scrunched in to the origin.

 More on generalized functions.  Identify the function $u(x)$ with
 the linear functional
 \[
    \psi\mapsto \int u(x)\psi(x)\, dx
 \]
 where $\psi$ (the ``test function'') ranges over
 $C^{\infty}_c(\RR)$.  Thus, when we talk about $e^{\frac{i}{2}
 ax^2}$ we should think about how it acts on test functions.  Well
 if $\psi$ is supported away from the origin, $e^{\frac{i}{2}
 ax^2} \psi$ integrates to something very small.  You can check
 that if $\supp \psi$ doesn't contain the origin, then $\int e^{\frac{i}{2}
 ax^2}\psi = o(\frac{1}{a^N}$ for $N>0$.  If the origin is in the
 support of $\psi$, then we can check that
 \[
    \sqrt{a} \int e^{\frac{i}{2} ax^2}\psi(x)\, dx \to \sqrt\pi (sign\
    a)\psi (0)
 \]

 ``Geometric WKB prescription''  Identify the graph of
 $\im(dS)\in \lag(T^*X)$ with the generalized function(s)
 $const\cdot e^{iS}$.  This const also eliminates the problem that
 we can change $S$ by a constant.  Then this correspondence
 extends in a reasonable way, associating to more general
 lagrangian submanifolds on $T^*X$ distributions on $X$.

 We think of the lagrangian submanifold as a state.  This is the
 beginning of the association of classical states and quantum
 mechanics (or in general, analysis on $X$ with geometry on
 $T^*X$).

 Another remark:  in linear algebra, if you had $V\oplus \bar V$,
 you could sometimes identify it with an $E\oplus E^*$.  In this
 way, you get some identification of $\aut V$ with $Symm(E)$.  Now
 let's try it on manifolds.  Suppose you could identify $\bar
 M\times M$, as a symplectic manifold, with $T^*X$.  Then we get
 some identification between $\aut M$ and generalized functions on
 $X$, $C^{\infty}(X)$.  The function associated to an automorphism
 is called the \emph{generating function} of the morphism.  In
 fact, there are different ways to pick $T^*X$, or the
 identification.  This gives different flavors of generating
 functions.

 \[
  \begin{pspicture}(-.6,-.6)(8,2)
  \psline(-.3,0)(-.3,2)\rput(-1,1){$M$}
  \psline(0,-.3)(2,-.3)\rput(1,-.6){$\bar M$}
  \pspolygon(0,0)(0,2)(2,2)(2,0)
  \psline(0,0)(2,2)\rput(1.2,1.5){$\Delta$}
  \pscurve(0,0)(.5,.7)(1.5,1.2)(1.8,1.9)(2,2) \rput(1.5,.9){$f$}
  \pscurve{->}(2.2,2)(2.5,1.9)(3,1)(3.8,1)
  \pspolygon[linecolor=white,fillstyle=vlines,hatchangle=0,hatchcolor=gray](4,0)(4,2)(6,2)(6,0)
  \rput(6.5,2){$T^*X$}
  \psline(4,-.3)(6,-.3)\rput(5,-.6){$X$}
  \psline(4,1)(6,1)
  \rput(7.2,1.1){zero section}
  \pscurve(4,1)(4.5,1.2)(5.4,.8)(5.8,1.1)(6,1)
  \rput(5.5,.5){$dS$}
  \end{pspicture}
 \]

 If you identify the diagonal on $\bar M\times M$ with the zero
 section on $T^*X$, then you get an identification of $X$ with
 $M$.  And then a symplectomorphism close to the identity gives a
 closed 1-form.  The points on the diagonal (fixed points of $f$)
 go to points on the zero section (critical points of $S$).

 Take the example where $M=S^2$ with the usual symplectic
 structure.  There is a theorem like the Lefshetz fixed point
 theorem that an automorphism close to the identity, then there
 are two fixed points (with multiplicity).  Now suppose you have
 an area-preserving map on $M$ (i.e. a symplectic map).  Then this
 would correspond to some $dS$, where $S$ is a function on $S^2$.
 The section $S$ must have two distinct critical points (a max and
 a min).  Thus, an area-preserving map close to the identity has
 at least two (geometric) fixed points.  This is the simplest case
 of the \emph{Arnol'd conjecture}: if you have a symplectomorphism
 close to the identity, then the number of fixed points is the
 same as the number of critical points of some function.

 The technology for solving the Arnol'd conjecture developes into
 ways of analyzing intersections of lagrangian submanifolds, which
 is what lead to \emph{Floer homology}.
