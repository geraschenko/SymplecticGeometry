 \stepcounter{lecture}
 \setcounter{lecture}{13}
 \sektion{Lecture 13 - Momentum maps and Symplectic reduction}

 Next meeting is in 450C Moffit Library.

 There are some nice ways to simplify your life.  One way is to
 find symmetries, and the other is to find invariants of motion.
 If you can do one, you can do the other.  This follows from
 \[
    X_H F=\{F,H\} = -\{H,F\} = -X_F H
 \]
 If one is zero, so is the other.  In general, when you take a
 quotient or look at a subspace (on which something is constant),
 you lose the symplectic structure.

 Lets take $\RR^4$ with $(q^1,q^2,p_1,p_2)$ and the standard form
 $dq^i\wedge dp_i$.  We can take, for example
 \[
    H(q^2,p_1,p_2)=\frac{1}{2}g^{ij}(q^2)p_ip_j+V(q^2)
 \]
 We have that $\{H,p_1\}=0$.  Let's impose the condition
 $p_1=c=$const.  Now we are down to three dimensions.  On
 $\{p_1=c\}$, we have the coordinates $(q^1,q^2,p_2)$, and the
 induced form, $dq^1\wedge dq^2$, is degenerate.

 On the other hand, $\pder{H}{q^1}=0$, and $\L_{\pder{}{q^1}}\w=0$, so we can pass to
 the quotient by the flow of $\pder{}{q^1}$.  Now we have
 coordinates $(q^2,p_1,p_2)$.  But we don't have a way of pushing
 forward the form (you cannot have a non-degenerate form be the
 pullback of a lower-dimensional form).  But you can push the
 poisson structure down.  $\pi=\pder{}{q^i}\wedge \pder{}{p_i}$
 pushes down to $\pder{}{q^2}\wedge\pder{}{p_1}$.  And
 $\{q^2,p_2\}=0$, and $\{f,p_1\}=0$ for all $f$ ($p_1$ is called a
 Casimir function).  This is a degenerate poisson structure.

 In both cases, we get some degeneracy.  How can we get back to
 symplectic geometry?

 \underline{Presymplectic}: (the first case)  you can still find a
 hamiltonian vector field.  So $X_H\lrcorner \w=dH$, or $\tilde
 \w(X_H)=dH$.  We have that $\tilde\w(\pder{}{q^1})=dp^1=0$,
 $\tilde\w(\pder{}{q^2})=dp^2$, and
 $\tilde\w(\pder{}{p_2})=-dq^2$.  $H=H(q^2,c,p_2)$, so it doesn't
 have any $dq^1$ in it, so $dH=\pder{H}{q^2}dq^2 +
 \pder{H}{p_2}dp_2$, so we get
 \[
    X_H = -\pder{H}{q^2}\pder{}{p_2} + \pder{H}{p_2}\pder{}{q^2} +
    (anything)\pder{}{q^1}
 \]
 We have an $H$ which happens to be in the image of $\tilde \w$,
 but the hamiltonian vector field is not well defined because of
 that $anything$.  We can resolve this problem by passing to the
 quotient:
 \[
    \{p_1=c\}/\underbrace{(\text{flow of }X_{p_1}=\pder{}{q^1}
    )}_{\text{ker of }\tilde \w}
 \]
 Now we have coordinates $(q^2,p_2)$, with form $dq^2\wedge dp_2$,
 and a ``reduced hamiltonian'', $H(q^2,c,p_2)$.  We get dynamics
 on this $\RR^2$, which is the projection of what we had before.

 \[\begin{pspicture}(0,-.8)(7,3)
   \psline[linecolor=lightgray](2,0)(2,2)(0,2)(0,0)(2,0)(3,1)(3,3)(2,2)
   \psline[linecolor=lightgray](0,2)(1,3)(3,3) \rput(3.5,3){$\RR^4$}
   \pspolygon[fillstyle=hlines,hatchangle=0,hatchcolor=gray](.5,.5)(.5,2.5)(2.5,2.5)(2.5,.5)
        \rput(1.5,2.7){$p_1=c$}
   \psline{->}(0.3,-.2)(1.3,-.2) \rput(.8,-.6){${}_{\pder{}{q^1}}$}
   \multirput(.5,1)(.4,0){5}{\psline{->}(0,0)(.3,.2)}
   \multirput(.5,1.5)(.4,0){5}{\psline{->}(0,0)(.3,.35)}
   \psline{->}(3.5,1.5)(5.3,1.5)
   \psline(6,.5)(6,2.5) \rput(6.5,2.5){$\RR^2$}
   \psline{->}(5.9,1)(5.9,1.2)
   \psline{->}(5.9,1.5)(5.9,1.8)
   \end{pspicture}
 \]

 $X_H$ tangent to the slice $p_1=c$ and invariant along
 $\pder{}{q^1}$, then we look at the quotient.

 For our hamiltonian, we end up with
 \[
    H = \underline{\frac{1}{2}g^{22} (q^2) p_2p_2} + \overbrace{g^{12}(q^2)cp_2}^{\text{vector potential}} +
    \underbrace{\frac{1}{2} g^{11}(q^2)c^2 + V(q^2)}_{\text{new potential}}
 \]
 This is pretty interesting, we again get something quadratic in
 the momentum plus an extra term (the \emph{vector potential}), plus a new
 potential.  So our reduces system behaves as though there was a
 magnetic field or something.  For example, when you have a
 rotating system, you get a force perpendicular to the direction
 of motion (coriolis force).

 On the other side, we have $\pi=\pder{}{q^2}\wedge \pder{}{p_2}$,
 so we get $\tilde\pi (dH)=X_H$, a well defined vector field.  We
 have that $X_HF = \{F,H\}$.  In particular, if $F$ is Casimir
 function, it is a constant of motion for any hamiltonian flow.
 This picture is dual to the previous picture.  Before, we could
 only solve for the hamiltonian if it was invariant along ***, and
 it wouldn't be well defined.  Here, we also started with a four
 dimensional space, then we pass to the quotient.  There is then a
 special direction (the $p_1$ direction).  Each of the planes
 $p_1=c$ are invariant under any hamiltonian flow, so these
 submanifolds have an induced symplectic structure (these are
 called the symplectic leaves).

 \[\begin{pspicture}(0,-.8)(7,3)
   \psline(2,0)(2,2)(0,2)(0,0)(2,0)(3,1)(3,3)(2,2)
   \psline(0,2)(1,3)(3,3) \rput(3.5,3){$\RR^4$}
   \psline{->}(0.3,-.2)(1.3,-.2) \rput(.8,-.6){${}_{\pder{}{q^1}}$}
   \psline{->}(3.5,1.5)(5.5,1.5)
   \pspolygon[fillstyle=vlines,hatchangle=0,hatchcolor=gray,hatchsep=2pt](6,0)(7,1)(7,3)(6,2)(6,0)
   \psline{->}(6.2,0)(7,.8) \rput(6.9,.3){$p_1$}
   \psline{->}(7.2,1.2)(7.2,2.8) \rput(8,2){$(q^2,p_2)$}
   \end{pspicture}
 \]

 We get exactly the same flow as before.  So there are two ways to
 do this.  Start with a big phase space, then pass to a
 presymplectic subspace, then take a quotient to get something
 symplectic.  Or you can first take a quotient to get a poisson
 structure, then take a subspace to get something symplectic.

 This is an example of symplectic reduction, but we've done the
 special case where the symmetry group is $\RR$.  This is more or
 less general.  If $f\in C^\infty(M)$, with $M$ symplectic, and if
 $(df)(x)\not=0$, then near $x$, there are canonical coordinates
 $(q^i,p_i)$, in which $f=p_1$.  Thus, locally, this is the
 general case.  Sometimes more interesting things happen.  For
 example, the orbits of the flow might close up on themselves,
 which is exactly what happens in rotational motion.

 \subsection*{Moment(um) Map(ping)s}

 We have a symplectic manifold $(M,\w)$ (though there are versions
 for the presymplectic and the poisson cases (and dirac as well)),
 and we have a lie group $G$ acting on $M$ symplectically.  So we
 have $g\mapsto g_M$ with $(gh)_M-g_m\circ h_M$.  This induces a
 lie algebra action.  Namely, if you take $v\in \g$, you get a
 vector field $v_M\in \chi(M)$ given by
 \[
    v_M= \der{}{t}{\Big |}_{t=0}(\exp \, tv)_M.
 \]
 It turns out that $(v+w)_M=v_M+w_M$, $(av)_M=av_M$, and
 $[v,w]_M=-[v_M,w_M]$.  If we have
 \[\xymatrix{
 G \ar[r]& Diff(M)&  g \ar@{|->}[r] & g_M\\
 }\]
 then we get a map $G\to \aut(C^\infty(M))$, $g\mapsto g_M^*$,
 then $\hat v_M=\der{}{t}{\Big |}_{t=0} ((exp\, tv)_M)^*$ as a
 derivation of functions, and the bracket of vector fields is just
 the bracket of operators.  From this point of view, we have a map
 from one lie group to another, so it should induce a lie algebra
 homomorphism.  But $g\mapsto g_M$ is a homomorphism and
 $g_M\mapsto g_M^*$ is an anti-homomorphism, which is why you get
 a minus sign.

 \[\xymatrix{
 \g \ar[r]^(.4){antihom} \ar@{.>}[d]_{\J} & \chi(M,\w)\\
 C^\infty(M)\ar[ur]_{antihom}
 }\]
 If $H^2(\g,\RR)\simeq \g/[\g,\g]$ vanishes, you get a lift $\J$,
 the comomentum map.  It is not unreasonable to require $\J$ to be
 a lie algebra homomorphism.  Alternatively, since $G$ acts on
 $\g$ by the adjoint representation and it acts on $C^\infty(M)$
 by $g_{C^\infty(M)}=(g^{-1}_M)^*$, we could require that $\J$
 $G$-equivariant.

 \begin{proposition}
   $\J$ is a lie algebra homomorphism if and only if $\J$ is
   equivariant for the action of the identity component of $G$ (if
   and only if $J$ is $Ad^*$-equivariant for the action of the
   identity component of $G$).
 \end{proposition}
 \begin{proof}
   Look in the book.
 \end{proof}

 There is another nice way of understanding all this, which was
 introduced independently by several people.  Kostant, Souriau,
 and Smale.  Siriau is the most generat.  Kostant only looked at
 transitive actions. Smale only looked at cotangent bundles.  They
 all reformulated the comomentum map.  To $\J:\g\to C^\infty(M)$
 (any linear map, at this point), we associate $J:M\to \g^*$ by
 $J(x)(v)=\J(v)(x)$ for $x\in M,v\in \g$.  We are really looking
 at a function of two variables: $M\times\g \to \RR$.  We can
 think of it as $M\to g^*$ or $\g\to C^\infty(M)$.  $\J$ is
 equivariant if and only if $J$ is equivariant.  What are the
 group actions in question?  We have a group acting on $M$, and we
 have $G$ acting on $\g^*$ given by $g_{\g^*}=(Ad_{g^{-1}})^*$
 (this is called the coadjoint action).  Some people will write
 $Ad^*_g$, so you have to watch out.

 A \emph{hamiltonian action} (sometimes called a \emph{poisson
 action}) of $G$ on $(M,\w)$ is a symplectic action together with
 an equivariant momentum map $J:M\to \g^*$.  This isn't completely
 standard.

 Where are we going?  One place is a generalization of reduction
 from the case of one-parameter groups to more general groups

 \begin{theorem}[Symplectic Redcution Theorem, Marsden-Weinstien,Meyer]
   Say $(M,\w,G,J)$ a hamiltonian action, and $\mu \in \g^*$
   quasi-regular [regular] value, then $J^{-1}(\mu)$ is invariant
   under the action of $G_\mu=\{g\in G|Ad^*_{g^{-1}}\mu=\mu \}$,
   the coadjoint isotropy, and the orbits of $G_\mu$ on
   $J^{-1}(\mu)$ form a regular foliation. Let $i_\mu^*\w$ be the
   pullback of $\w$ to $J^{-1}(\mu)$, where
   $J^{-1}(\mu)\stackrel{i}{\hookrightarrow} M$. At each $x\in
   J^{-1}(\mu)$, $\ker_x\widetilde{i_\mu^*\w} = T_x(G_\mu x) = \{
   v_M(x)|v\in \g_\mu\}$.
 \end{theorem}
 Where if $M\xrightarrow{\phi}N$, $\mu\in N$ is
 \emph{quasi-regular} if $\phi^{-1}(\mu)\subseteq M$ is a
 submanifold, and for any $x\in \phi^{-1}(\mu)$, the inclusion
 $T_x\phi^{-1}(\mu)\hookrightarrow \ker(T_x(\phi))$ is an
 equality.  This simplest example of a non-quasi-regular value is
 when $\phi:x\to x^2$ on $\RR$, $x=0$ is not quasi-regular.  The
 implicit function theorem tells you that regular values are
 quasi-regular.

 This gives us the following picture:

 \[\begin{pspicture}(0,-2.5)(9,3)
   \psline[linecolor=lightgray](2,0)(2,2)(0,2)(0,0)(2,0)(3,1)(3,3)(2,2)
   \psline[linecolor=lightgray](0,2)(1,3)(3,3) \rput(3.5,3){$M$}
   \pspolygon[fillstyle=hlines,hatchangle=0,hatchcolor=gray](.5,.5)(.5,2.5)(2.5,2.5)(2.5,.5)
        \rput(1.7,2.7){${}_{J^{-1}(\mu)}$}
   \psline{->}(3.5,1.5)(5.3,1.5) \rput(4.4,1.7){$\pi_\mu$}
   \psline(6,.5)(6,2.5) \rput(6.5,3){$J^{-1}(\mu)/G_\mu = M_\mu$}
   \psline{->}(3,.2)(2.2,.8) \rput(3.8,0){$G_\mu\text{ orbits}$}
   \psline{->}(1.5,-.4)(1.5,-1.5) \rput(1.2,-1){$J$}
   \psline(1,-2.5)(2,-1.5) \rput(2.3,-1.4){$\g^*$}
   \psdots(1.5,-2) \rput(1.65,-2.2){$\mu$}
   \end{pspicture}
 \]

 If $J^{-1}(\mu)/G_\mu$ is a manifold, with $\pi_\mu$ a
 submersion, then there is a (unique) symplectic form $\w_\mu$ on
 $J^{-1}(\mu)/G_\mu$ such that $\pi_\mu^*(\w_\mu)=i_\mu^*(\w)$.
 Then $(M_\mu,\w_\mu)$ is called the reduction of $M$ at $\mu$.
