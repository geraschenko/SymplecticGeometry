 \stepcounter{lecture}
 \setcounter{lecture}{22}
 \sektion{Lecture 22 - Geometric Quantization}

 The first draft of the paper is due next Tuesday!

 We have $(M,\w)$ a symplectic manifold, and a prequantization is a circle bundle
 $(Q,\phi)$ over $M$ whose curvature is $\w$.  A necessary condition is that
 $[\w]$``$\in$''$H^2(M,2\pi\ZZ)$.  If it exists, it is unique up to isomorphism and
 tensor product with a flat $U(1)$ bundle-with-connection.  These correspond to
 elements of$H^1(M,U(1)) = \hom(H_1(M,\ZZ),U(1)) = \hom(\pi_1(M),u(1))$.  If you look
 at the set of all such pre-quantization bundle, it is a torus (at least if
 $H_1(M,\ZZ)$ is free, otherwise a torus times finite cyclic group).

 Once we choose such a thing, we get $\H^n\subseteq C^\infty(M)$ which is the
 $\sqrt{-1} n$ eigenspace of $\xi$ (the generator of the $U(1)$ action).  There is a
 sense in which $n\to \infty$ is like $\hslash\to 0$.  If we take $\H^n_{(M,\w)}\simeq
 \H^1_{(M,n\w)}$.  Recall that $\H^1_{(M,\w)}\simeq \Gamma(E)$, where $E$ is a complex
 line bundle.  $\H^n_{(M,\w)}\simeq \Gamma(E^{\otimes n})$.  We know that the chern
 class of $E^{\otimes n}$ is $n$ times that of $E$, so it is like taking $n\w$ in
 place of $\w$.  $\w$ has units of action, so you have to divide by the unit of
 action, which is $\hslash$. Working in $Q$, we can understand what happens as $n\to
 \infty$.

 Recall also that given $f\in C^\infty(M)$, we get $Y_f^n = -\hat X_f + inf$ (or
 $-\hat X_f +f\pder{}{\theta}$ as a vector field).  Notice that in the classical limit
 ($n\to \infty$), it looks more and more like pointwise multiplication ($Y_f^n$ looks
 like $f$).

 \underline{Quantomorphisms}: A quantomorphism is an automorphism of $(Q,\phi)$.  Note
 that anything preserving $\phi$ preserves $\xi$.  $Quant(Q,\phi)\to Symp(M,\w)$
 because $\w$ can be gotten from $\phi$.  What are the quantomorphisms which give you
 the identity?  It is just the $U(1)$ action (all the fibres have to rotate with the
 same speed).  You can check that the $Y_f$ are all the quantomorphisms.  You can
 check this by breaking it up into vertical and horizontal parts.
 \[
    1\to U(1)\to Quant(Q,\phi) \to Symp(M,\w) \xrightarrow{?}
 \]
 On the lie algebra level, $Quant(Q,\phi)$ is $C^\infty(M)$.  In $Symp(M,\w)$, we have
 $\chi(M,\w)$, the symplectic vector fields.  In $U(1)$, we have $H^0(M,\RR)$ ($\simeq
 \RR$).
 \[
    0\to H^0(M,\RR) \to C^\infty(M) \to \chi(M,\w) \to H^1(M,\RR) \to 0.
 \]
 If we have a group $G\hookrightarrow Symp(M,\w)$, and we want to lift it to a group
 of quantomorphism,  this is related to lifting the map $\g\to \chi(M,\w)$ to $\g\to
 C^\infty(M)$, which is the story of momentum maps.  If we can lift $G$, then $\g$ has
 to lift, so we have to have a hamiltonian action.  If we have a hamiltonian action,
 then we can lift $\g$, then does it follow that $G$ lifts?  If we have a homomorphism
 of lie algebras and we want to lift it to a homomorphism of lie groups, we can do it
 locally.  We can only do it globally if $G$ is simply connected.  The problem is that
 if you have a closed 1-parameter subgroup, when you lift it, it might not close up.

 Example: $M=S^2$, with the circle action (by rotation about the vertical axis).  We
 use coordinates $(I,\psi)$ on the sphere ($I$ is height).  Then $\w=d\psi\wedge dI$.
 If $-\sigma \le I\le \sigma$, then the total (symplectic) area is $4\pi \sigma$.  So
 the is prequantizable if and only if $\sigma \in \frac{1}{2}\ZZ$ (because the
 cohomology class has to be in $2\pi\ZZ$).  The quantization is unique because it is
 simply connected (the quantization is the hopf fibration).  $SU(2)$ and $SO(3)$ act
 on $S^2$ is the usual way.  Does the action lift to the quantization? Let's look at
 rotations around the $z$-axis.  This rotation is generated by the function $I$ and
 has period $2\pi$.  What happens when we try to lift the action?  We have to look at
 the vector field $Y_I = -\hat X_I + I\pder{}{\theta}$.  Above a critical point of
 $I$, $Y_I$ is just $I\pder{}{\theta}$.  Let $N$ and $S$ be the north and south poles.
 Over $N$, we have that $Y_I=\sigma\pder{}{\theta}$ and over $S$, $Y_I =
 -\sigma\pder{}{\theta}$.  So the period of the action is $2\pi/\sigma$ upstairs.
 Remember that the action on the two-sphere has period $2\pi$, so we must require that
 $2\pi$ is an integer multiple of $2\pi/\sigma$, so only the integer $\sigma$s allow
 the action of $SO(3)$ to lift.  This exactly corresponds to the fact that the
 irreducible representations of $SU(2)$ are parameterized by spin, which can be half
 integer, but only the integer spins give irreducible representations of $SO(3)$.  If
 $\sigma=1/2+k$, then when we go half way around $SU(2)$, we go all the way around
 $S^2$, but only half way around upstairs.  If you let it act on the space $\H^{\pm
 1}$, you get multiplication by $-1$.  We have $SU(2)$ acting on its coadjoint orbits,
 we have an infinite dimensional representation, which is not irreducible.  We cut it
 down to size by polarization.

 \subsection*{Polarization}
 Integrable Lagrangian sub-bundle $F\subseteq TM$.  The leaves gives us a foliation,
 which we sometimes call the polarization.  The example above is not quite a foliation
 because we have singularities at the two poles.  Let's say we work away from this
 badness.  We have $(Q,\phi)$ sitting over $M$ (via $p:Q\to M$).  If we restrict $Q$
 to one of the leaves $\L$ of our foliation, we have $p^{-1}\L$.  The curvature is given
 by the pullback of $\w$, but since $\L$ is lagrangian, this is zero.  So $Q$ is flat
 on the leaves of the foliation.  $\ker\phi$ is a contacts structure, and when you
 restrict to the leaves, you have a closed 1-form.  Over each $\L$, there is the
 holonomy map $\pi_1 \L\to U(1)$, which is in $H^1(\L,U(1))$.

 \def\BS{\mathcal{BS}}

 Let's look at $\H^{-1}_{(M,\w)}$``$\supseteq$''$\H^{-1}_{(M,\w,F)} = $ functions
 constant along the leaves of the lifted foliation given by $\hat F$.  When do we have
 such a function?  When we move along a fibre, it is just some value times
 $e^{i\theta}$.  If the holonomy is non-trivial, something would have to be zero.  So
 such functions are supported on leaves with trivial holonomy.  This set of leaves is
 called the ``Bohr-Sommerfeld set'' $\BS$.  If the leaves are simply connected, then great
 ... we don't have to worry about this condition.

 Notice that if you fix a leaf which is in the Bohr-Sommerfeld set, and you look at
 the functions supported on that leaf which are in $\H^{-1}$, and which are constant
 along leaves, then the function is determined by one value in the leaf.  We have that
 $\BS(F)/F\subseteq M/F =$ the leaf space of $F$.  Upstairs, we have
 $p^{-1}(\BS(F))/\hat F$, with kernel $U(1)$ (we can identify the different circles by
 parallel translation).  We've reduced ... instead of looking at all functions on $Q$,
 we restrict to $\BS$, and we are constant on the leaves.

 In our example, what is $\BS$?  We want to know the holonomy of a leaf.  If
 $\w=-d\alpha$ in some area of your manifold, and assume $M$ is simply connected (for
 simplicity), then we can take $Q$ to be $M\times U(1)$ with $\phi = d\theta -\alpha$.
 If we take a loop in $M$, we can define the holonomy, but it wont be homotopy
 invaiant.  The holonomy of some loop $\gamma$ is just $\int_\gamma \alpha$.  When we
 lift a loop, $\phi=0$, so $d\theta = \alpha$.  Then to find the holonomy, we
 integrate $d\theta=\alpha$.  If $\gamma$ bounds a surface $\Sigma$, then this is
 $\int_\Sigma \w$ by Stokes theorem.  Notice that this is constant with other things.
 If we have two different surfaces bounded by $\gamma$, then the difference is
 required to be a multiply of $2\pi$.  We should really take the holonomy to be
 $e^{\int_\gamma \alpha}$.  The $\BS$ set is the set where the holonomy is trivial (a
 multiple of $2\pi$).  So $\BS$ consists of the leaves at height $\sigma, \sigma -1,
 \dots, 1- \sigma, -\sigma$.  So there are $2\sigma +1 $ leaves (if you allow us to
 count the poles as leaves.  Over each of these $2\sigma+1$ points, we have a complex
 line.  Something is sections of this bundle.  So we get a $2\sigma +1$ dimensional
 space.  How does the circle group act on this space?  Take the generator
 $\pder{}{\psi}$.  Take an element which is 1 on the leaf corresponding to $s$ for
 $-\sigma \le s \le \sigma$ an integer.  Something in this space is killed by the
 horizontal distribution, so $\xi$ acts by $-\sqrt{-1}s$.  This operator of lifting is
 called spin along the $z$-axis.  In the language of representation theory, the
 circles are maximal tori, and the $-\sqrt{-1}s$ are the weights.

 What if we change the foliation ... do we get the same representation or not?  Let's
 use the foliation with respect to the east pole and the west pole.  What can we say
 about spin around the $z$-axis?  It doesn't leave the polarization invariant.
 Geometric quantization works very nicely to build representations of groups which
 leave a polarization invariant.  In this case, the action descends to an action on
 $\BS$.  But in most examples, the groups don't leave the polarization invariant.  Two
 ways out.  First way is to deal with non-invariant polarizations ... this is called
 the Blattner-Kostant-Sternberg pairing or the projection method or Toeplitz
 quantization.  The other approach is to widen your notion of what a polarization is.
 There is no polarization invariant under the action of $SO(3)$, so we allow complex
 polarizations (a complex structure compatible with the symplectic structure ... a
 K\"ahler structure).

 Example: $M=T^*X$ with $Q=T^*X\times U(1)$, $\phi=-\alpha + d\theta$, where $\alpha$
 is the liouville form, $F$ is the foliation by the fibres.  In this case, everything
 is trivial, and you find that the fibres are simply connected so $\BS$ can be
 identified with the base manifold, so we just have functions on the base $X$.  The
 functions preserving the fibration are the ones which are affine on the fibres.
