 \stepcounter{lecture}
 \setcounter{lecture}{26}
 \sektion{Lecture 26 - Poisson Geometry}

 A poisson structure is$\pi \in \wedge^2 TM$, we we get $\tilde \pi: T^*M\to TM$, then
 define $\{f,g\}:=\pi(df,dg)$.  $X_f=\{\cdot, f\}= \tilde \pi(df)$.  To figure out the
 right signs, we want
 \begin{align*}
   \w=dq\wedge dp & \tilde \w(\pder{}{q}=dp\\
   \pi=\pder{}{q}\wedge\pder{}{p} & \tilde \pi^{-1}=\tilde \w & \tilde
   \pi(dp)=\pder{}{q}
 \end{align*}
 so $\alpha(\tilde \pi(\beta))=\pi(\alpha,\beta)$, $\tilde\w(x)(y)=\w(x,y)$.  Then we
 also have that $[X_f,X_g]=X_{\{f,g\}}$, so hamiltonian vector fields are closed under
 $[,]$.

 In the symplectic case, if $X,Y$ are symplectic vector fields (i.e.
 $\L_X\w=\L_Y\w=0$), then the bracket, $[X,Y]$, is hamiltonian.  However, this is not
 true in the Poisson case.  Note, by the way, that the zero poisson structure is a
 poisson structure.

 \underline{Examples}
 \begin{itemize}
 \item[(a)] $\pi=0$.
 \item[(b)] $\pi=\phi(x,y)\pder{}{x}\wedge \pder{}{y}$ for any $\phi$ in $\RR^2$.
 \item[(c)] $\pi = \frac{1}{2}c_k^{ij} x^k\pder{}{x^i}\wedge \pder{}{x^j}$ on $\g^*$.
 \end{itemize}

 What do poisson structures look like?  Locally, symplectic structures all look the
 same (Darboux).  $\tilde\pi (T^*M)\subseteq TM$ is not a sub-bundle in general, it is
 a distribution.  In case b, the dimension goes to zero when $\phi$ vanishes.  The
 natural sections of $\tilde\pi(T^*M)$ are $\tilde \pi($a 1-form$)$.
 \begin{theorem}
   $\tilde\pi (T^*M)$ is integrable in the sense that $M$ is a disjoint union of
   connected integral manifolds.
 \end{theorem}
 a: integral manifolds are points\\
 b: integral manifolds are components of the open set where $\phi\neq 0$.\\

 A point $m\in M$ is \emph{regular} if rank$(\tilde \pi)$ is constant on a
 neighborhood of $m$.  $\pi$ induces on each leaf of $\tilde \pi(T^*M)$ a symplectic
 structure because $T_m^*M/\ker \tilde \pi\simeq \tilde \pi (T_mX)$.  The leaves are
 the \emph{symplectic leaves} of $(M,\pi)$.

 [e.g. let $\pi=0$ for $y\le 0$ and non-zero elsewhere]

 On a surface, the generic situation is this: you have curves where $\pi=0$, and a
 bunch of open symplectic leaves.  The thesis of O.~Radko classifies all of these
 structures.

 The $f$ such that $X_f=0$ are \emph{Casimir functions}.  These are exactly the
 functions constant along leaves.

 \underline{Local Structure}:
 \begin{theorem}[Splitting theorem]
   There are local coordinates $q^i,p_i,y^j$ $1\le i\le k$, $1\le j\le r$ such that
   \[
    \pi = \pder{}{q^i}\wedge \pder{}{p_i} - \pi^{rs}(y)\pder{}{y^r}\pder{}{y^s}
   \]
   where $\pi^{rs}(0)=0$.
 \end{theorem}
 So you have a symplectic manifold, and a bunch of transverse guys with poisson
 structures.  You can flow along a hamiltonian vector filed, preserving poisson
 structure, to get between transverse sections.  So near a leaf, there is a product
 structure, but these things can glue together funny.  See Vorobev, Davis-Wade, and
 somebody in Belgium.

 Going back to the local question.  It is enough to classify the $\pi^{rs}$.  Consider
 a tweak of example c:
 \[
 \pi = \frac{1}{2}c_k^{ij} y^k\pder{}{y^i}\wedge \pder{}{y^j} + O(y^2) \tag{$\ast$}
 \]
 is the most general version.

 If you just write $\pi=\pi^{ij}\pder{}{y^i}\wedge\pder{}{y^j}$, you don''t get a
 poisson structure in general, you need $[\pi,pi]=0$ for the Schouten-Nijenhuis
 bracket. The $c$ term something lie algebra.

 Something ... the transverse leaves are lie algebras.  At a regular point, the rank
 can't change when you move around, so the transverse structure (the $\pi^{rs}$s) is
 zero, and the $y$'s are local Casimirs. This was done by Lie.

 How general are the linear structures?  Given a $\pi$ like $\ast$, do there esist new
 coordinates $z^i=y^i+O(y^2)$ such that $\pi = \frac{1}{2}
 c_k^ij\pder{}{z^i}\wedge\pder{}{z^j}$.  Sometimes you do.  But there are some
 theorems that say that if the lie algebra is complicated enough, you can linearize
 like this.

 Linearization theorems: We say a lie algebra $\g$ is [formally, smoothly,
 analytically] non-degenerate if any [formal,$C^\infty$,$C^\w$] poisson structure
 whose linearization is isomorphic to $\g^*$ is locally isomorphic to $\g^*$.
 \begin{theorem}[Arnol'd]
   If $\g=\{[x,y=y]\}$, i.e.~$\{x,y\}=y+O(x,y)^2$, then you can linearize in ``any
   category''.
 \end{theorem}
 \begin{theorem}[Weinstein]
 If $\g$ is semi-simple, you can formally linearize. $\g=\mathfrak{sl}_2(\RR)$, then
 not $C^\infty$.
 \end{theorem}
 \begin{theorem}[Conn]
   $\g$ semi-simple, analytic works, and $C^\infty$ only if $\g$ is also of compact
   type (Killing form definite, note just non-degenerate).
 \end{theorem}
 \dots, most due to Dufour or students of his, or people who work with him, like Wade,
 Nguyen Tien Zung, Monnier, Stolovich (holomorphic poisson), etc.  There are some
 other lie algebras (non-semi-simple) which still have this stability.

 Consider the structure
 \begin{align*}
   \{x,y\} &=0\\
   \{x,z\} &=ax+by\\
   \{y,z\} &=cx+dy\\
 \end{align*}
 the hamiltonian flow of $z$ is given by $\dot{x}=ax+by, \dot{y}=cx+dy$, which is an
 arbitrary linear differential equation. The flow of $y$ is $\dot x=\dot y=0,\dot
 z=-cx+dy$, so you can move up and down (almost always).  So the leaves will be
 cylinders on the trajectory on the plane.
