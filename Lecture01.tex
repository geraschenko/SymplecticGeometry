 \stepcounter{lecture}
 \setcounter{lecture}{1}
 \sektion{Lecture 1}

 \noindent Alan Weinstein\\
 office 825 Evans\\
 642-3518\\
 alanw@math.berkeley.edu\\
 OH: T 9:40-11, R 12:40-2\\
 \texttt{math.berkeley.edu/\~{}alanw}\\
 \texttt{bspace.berkeley.edu}

 \noindent GSI(sorta):Christian Bloman?

 orbit method in representation theory.

 symplectic geometry and algebraic geometry.  moment maps.
 Attiya.

 90s. conformal field theory.  Syberg-Witten theory.

 Poisson bracket ... lie algebra structure.  Leads to Poisson
 geometry.  Mechanics - Poisson brackets are classical limit of
 commutators.

 00s. Generalized complex geometry, which includes complex and
 symplectic geometry.

 \vspace{5mm}

 Weekly reading assignments - try to look at it before the week
 starts, and make questions.  Long list of weekly homework; you
 only have to hand in about three each week, due on the following
 Thursday.  Term paper - sample papers on website; the idea is to
 write a survey of the topic of your choice.  In October, we will
 have a session with librarian to learn how to search for stuff.  There
 is no final exam (no exams at all, in fact).

 There should be an email list on bspace, where you can send
 questions.  Until that is set up, send questions directly to Alan
 (answers sent to whole class unless otherwise specified).

 The book really is required.  There is a website for corrections.
 The author also has a paper on the arXiv which is kind of a
 sequel to the book. Please understand everything in the reading.

\vspace{5mm}

 \underline{Symplectic linear algebra:}

 Bilinear forms: $V$ vector space over a field (almost always $\RR$
 or $\CC$), but you could work over other fields or rings.

   A \emph{bilinear form} $B:V\times V\to k$ (sometimes to
   $\RR$ if $k=\CC$) is bilinear.  In the case $k=\CC$, we
   also have \emph{hermitian}\footnote{When some name becomes part
   of an adjective, you stop capitalizing}, which means $B(ax,by)
   = a\bar b B(x,y)$ for complex numbers $a$ and $b$, and
   $B(x,y)=\overline{B(y,x)}$. We can talk about \emph{symmetric}
   forms ($B(x,y)=B(y,x)$), \emph{skew symmetric}.

  In super algebra, symmetric and skew symmetric things are the
  same thing.  Hermitian and skew hermitian are closely related.

 Dualization:If $B$ is a bilinear form on $V$, it induces a linear map
  $\tilde B: V\to V^*$ given by $\tilde B(x)(y)=B(x,y)$.  This is
  the notion of \emph{dualization}.  We say $B$ is
  \emph{non-degenerate} if $\tilde B$ is an isomorphism.  Note
  that this gives a way to get $B$ from $\tilde B$, as well as the
  other way.

  Orthogonality: (symmetric, skew-symmetric, hermitian).  Two
  vectors are orthogonal if $B(x,y)=0$.  If $W\subseteq V$, we
  define $W^{\perp} = \{x\in V|B(x,y)=0\ \forall y\in W\}$.  In
  particular, $V^{\perp}=\ker \tilde B$, which is zero if $B$ is
  non-degenerate.  In the finite-dimensional case, the converse is
  true.  In the infinite-dimensional case, it turns out that many
  of the symplectic structures are \emph{weakly non-degenerate}
  (i.e. $\tilde B$ injective, but not surjective).  In the
  finite-dimensional case, non-degeneracy ensures
  $(W^{\perp})^{\perp}=W$

  Assume finite dimensional, non-degenerate.  Then $\dim W + \dim
  W^{\perp}=\dim V$.  In particular, if $W=W^{\perp}$, then $\dim
  W=\frac{1}{2} \dim V$ (and so $\dim V$ must be even).  Such $W$
  are called \emph{lagrangian} in the skew-symmetric non-degenerate case
  (which is exactly \emph{symplectic}).

  $W\subseteq V$ is \emph{isotropic} is $W\subseteq W^{\perp}$.
  In the skew-symmetric case, if $\dim W=1$, then $W$ is isotropic
  because $B(v,v)=-B(v,v)$, so $B(v,v)=0$.  Isotropy implies that
  $\dim W\le \frac{1}{2} \dim V$.  Any isotropic space is
  contained in a maximal isotropic space (which are lagrangians!).
  $W$ is \emph{coisotropic} if $W^{\perp}\subseteq W$.  That is,
  if $W^{\perp}$ is isotropic.

 By the way, symplectic implies even dimension!

  \underline{Examples:}
  \begin{itemize}
  \item[(a)] Let $E$ be any vector space, and let $V=E\oplus E^*$
  (this is even-dimensional).  Define $\W_{\pm}: V\times V\to k$
  by $\W_{\pm}((x,p),(x',p')) = \langle x,p'\rangle \pm \langle
  x',p\rangle$ (\footnote{Here, $\langle x,p\rangle = p(x)$.}).
  When is this non-degenerate?  $\tilde \W_{\pm}:V\to V^*$,
  sending $E\oplus E^*$ to $E^*\oplus E^{**}$ given by
  \[
    \W_{\pm}(x,p)(x',p') = p'(x)\pm p(x')
  \]
  so $\tilde \W_{\pm}(x,p) = (\pm p, i(x))$, where $i:E\to E^{**}$
  is the natural map.  In the finite-dimensional case, we can
  identify $i(x)$ with $x$ because $i$ is an isomorphism.  $E$ is
  called \emph{reflexive} if $i$ is an isomorphism, in which case
  $\W_{\pm}$ is non-degenerate.  When $E$ is finite-dimensional,
  note that $\W_-$ is symplectic!  This is in some sense
  \emph{the} example.

  In general, if you assume choice, $i$ is always injective, so
  $\W_{\pm}$ is always weakly non-degenerate.

  In many infinite-dimensional cases, we often require that maps
  be continuous (in some sense or another).  Then this gives some
  sort of topological dualization.

  \item[(b)] Let $E$ be a vector space.  $B$ any bilinear form on
  $E$, giving $\tilde B:E\to E^*$.  Look at the graph of $\tilde
  B$, which is a (linear!) subspace of $E\oplus E^*$.  We can ask,
  ``when is this graph lagrangian (w.r.t $\W_{\pm}$)?''  It has a chance because it
  is half dimensional.
  \begin{align*}
    \tilde \W_{\pm}((x,\tilde B(x)),(y,\tilde B(y))) &= \tilde
    B(y)(x) \pm \tilde B(x)(y)\\
    & = B(y,x)\pm B(x,y)
  \end{align*}
  So the graph is lagrangian for $\W_+\ [\W_-]$ if and only if $B$
  is skew-symmetric [symmetric].

  We think of lagrangian subspaces of $(E\oplus E^*,\W_{+[-]})$ as
  ``generalized'' skew-symmetric [symmetric] forms on $E$.

  From the graph of $\tilde B$, you can read off properties of
  $B$.  For example, $B$ is non-degenerate (in the finite
  dimensional case) if the graph doesn't intersect $E$.  A graph
  never intersects $E^*$.

  A lagrangian subspace of $(E\oplus E^*,\W_+)$ is called a
  \emph{Dirac structure} on $E$.  Special case is a skew-symmetric
  form.  And a special case of that is a symplectic structure.
  Note that $E$ and $E^*$ are themselves lagrangian.

  \item[(c)]  Operations on spaces with bilinear forms.
  $(V,B)\rightsquigarrow (V,-B)$ preserves type ... we often write
  this $V\to \bar V$, called the \emph{opposite symplectic
  structure}.

  Structure on $(V_1,B_1)\oplus (V_2,B_2)$ defined by
  $((x_1,x_2),(y_1,y_2))\mapsto B_1(x_1,x_2)+B_2(y_1,y_2)$.  We
  can add spaces with the same kind of structure.

  In $\bar V_1\oplus V_2$, given a linear map $L:V_1\to V_2$, when
  is its graph lagrangian?
 \[
    ((x,L(x)),(y,L(y)))\mapsto -B_1(x,y)+B_2(L(x),L(y)) = 0.
 \]
 The graph of $L$ is isotropic if and only if $L$ ``preserves
 `inner' product''.  It is lagrangian if and only if $L$ is an
 isomorphism of B-spaces (spaces with non-degenerate bilinear
 form).

 Lagrangian subspaces of $\bar V_1\oplus V_2$ as ``generalized''
 B-space isomorphisms of $V_1$ to $V_2$.

 In the symmetric case, these are isometries.  If the space is of
 the form $E\oplus E^*$ (which it always is), then we get a
 correspondence between isometries and skew-symmetric forms.  If
 $V_1=V_2$, isometries form the orthogonal group, and
 skew-symmetric forms form the lie group of that group!

 In the symplectic case, we will get a correspondence between some
 other things.

  \end{itemize}
