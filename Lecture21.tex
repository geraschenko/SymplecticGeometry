 \stepcounter{lecture}
 \setcounter{lecture}{21}
 \sektion{Lecture 21 - Prequantization}

 $C^\infty(M)$.  We are looking for a principal bundle $\RR/2\pi\ZZ \cong U(1)\to
 Q\xrightarrow{p} M$.  Let $\xi\in \chi(Q)$ generate the $U(1)$ action.  Then the
 connection is a form $\phi\in \W^1(Q)$ such that $\phi(\xi)=1$, $\L_\xi \phi =
 \xi\lrcorner d\phi = 0$.  Then $\ker \phi$ is a horizontal distribution.  $d\phi =
 p^*\w$.  That is, we want $\w$ to be the curvature of this connection.  Every $X\in
 \chi(M)$ has a unique horizontal lift $\hat X\in \Gamma(\ker \phi)$.  This lifting is
 not a lie algebra homomorphism.  In fact, $[\hat X,\hat Y] =
 \widehat{[X,Y]}+\phi([\hat X,\hat Y])\xi$.  Recall that \[d\phi (\hat X,\hat Y) =
 \hat X(\phi(\hat Y)) - \hat Y(\phi(\hat X)) - \phi([\hat X,\hat Y]).\]  Since these
 are horizontal, the first two terms are 0.  so $\phi([\hat X,\hat Y]) = -d\phi(X,Y)
 = -p^*\w([X,Y])$.  This gives you the vertical component of the bracket of two
 horizontal vectors.

 Given $f\in C^\infty(M)$, we attach to it $X_f$ (an anti-homomorphism), then we take
$Y_f := - \hat X_f + f\xi \in \chi(Q)$.  This vector field has the property that it is
the unique vector field whose horizontal component is $-\hat X_f$, and is a contact
vector field.  Let's check that
\begin{align*}
  \L(-\hat X_f + g\xi)\phi &= d(-X_f +g\xi)\lrcorner \phi + (-\hat X_f + g\xi)
  \lrcorner d\phi\\
  &= dg - \hat X_f\lrcorner d\phi\\
  &= dg - p^*(X_f\lrcorner \w)\\
  &= dg - p^*df
\end{align*}
which implies that $g = p^*f +const$.  Let's choose the constant to be zero.

 Now we compute
 \begin{align*}
   [Y_f,y_g] &= [-\hat X_f + f\xi, -\hat X_g + g\xi]\\
   &= [\hat X_f,\hat X_g] - [\hat X_f, g\xi] + [\hat X_g,f\xi]\\
   &= \widehat{[X_f,X_g]} - \w(X_f,X_g)\xi - (X_f\cdot g)\xi + (X_g\cdot f)\xi\\
   &= -\widehat{X_{\{f,g\}}} + (-\{f,g\} - \{g,f\} + \{f,g\})\xi\\
   &= Y_{\{f,g\}}.
 \end{align*}
 This correspondence ($f\rightsquigarrow Y_f$) is faithful ($Y_f=0 \Rightarrow f=0$).
 If we think of $f$ as a function of $q,p,\theta$.  We can write $\phi = d\theta -
 pdq$ in local coordinates.  Then $X_f = f_q\pder{}{p} - f_p\pder{}{q}$, so $\hat X_f
 = f_q\pder{}{p} - f_p\pder{}{q} + pf_p \pder{}{\theta}$, so
 \[
    Y_f = -f_q\pder{}{p} + f_p\pder{}{q} - pf_p \pder{}{\theta} + f\xi.
 \]
 So in particular, $Y_q = -\pder{}{p} + q\pder{}{\theta}$ and $Y_p = \pder{}{q}$.
 There is a geometric interpretation for this coordinate stuff.  Consider the case
 where $M=T^*X$, and then $Q= T^*X \times U(1)$ with $\phi = d\theta - \alpha$.  In
 this case, we find that for any $f\in C^\infty(X)$ (not $M$),
 $Y_f=-vert(df)+f\pder{}{\theta}$.  On the other hand, if we have a vector field,
 $\zeta$, which is linear on the fibers, then we can talk about $Y_\zeta =
 \pder{}{q}$, which is the cotangent lift of $\zeta$.

 Remember we are acting on functions of $q,p,\theta$, which is too many variables.  If
 we look at function independent of $p$, then $Y_p$ acts by differentiation, which is
 good, and $Y_q$ acts trivially.  If we add a $\theta$ dependence, then we get
 multiplication by $q$, which is what we want. What is the right kind of $\theta$
 dependence?

 $U(1)$ acts on $C^\infty(Q)$.  When you have a group acting on a vector space, you
 can break it up into irreducible representations.  The representations of $U(1)$ are
 classified by the eigenvalues of the generator $\xi$.  Typically, we have
 $f(e^{i\theta})$, and $\xi = \pder{}{\theta}$.  The eigenvalues are $n\in \ZZ$.  We
 can decompose $C^\infty(Q)$ into eigenspaces $\H_n$ of $\xi$.  In particular,
 $\H_0=p^*C^\infty(M)$.  On $\H_n$, we find that ``$\pder{}{\theta}=in$''.  Let
 $n=-1$.  Then $Y_q = -\pder{}{p} - iq$, $Y_p = \pder{}{q}$.  These are vector fields
 on a manifold, which have unitary flows, whose derivatives are skew.  To get
 hermitian operators, look at $iY_q = -\pder{}{p} + q$, $iY_p = i\pder{}{q}$.  If we
 now let these act on functions independent of $p$, then the operator corresponding to
 $q$ is multiplication by $q$ and that of $p$ is $i\pder{}{q}$, just like we wanted.
 This works whenever we have a circle bundle.

 In the special case where $Q=M\times U(1)$, then $u\in C^\infty(Q)$ can be written as
 $\sum a_n(x) e^{in\theta}$, where the $a_n$ are the fourier coefficients.  Then
 $\xi(u) = \sum in a_n(x)e^{in\theta}$.  For $u\in \H_n$, we have
 $u=a_n(x)e^{in\theta}$.  For $n=-1$, $u=a_{-1}e^{-i\theta}$, which is just a function
 of $x$.  This is in the case of the trivial bundle.  In general, we don't have a
 trivial bundle.  $\H_n$ is some vector space, but there is more structure.  Locally,
 an element is just a complex function on the base.  If we use pointwise
 multiplication, $\H_n\H_m\subseteq \H_{n+m}$ because $\xi$ is a derivation.  In
 particular, $C^\infty(M)\cong \H_0$, and $\H_n$ is a module over it ... it is a
 locally free module of rank 1.  So $\H_n$ is a line bundle.  We can identify the
 fibres.  We have $Q\xrightarrow{p} M$.  For $x\in M$, we define
 $\H^n_x$\footnote{$\H^n=\H_n$.} to be $\{u\in C^\infty(p^{-1}\{x\}) | \xi u =inu\}$.
 This is a complex line, so $E^n:=\bigcup H^n_x$ is a complex line bundle, so
 $\H_n=\Gamma(E^n)$.  We can also say that $E^n\cong (E^1)^{\otimes n}$, where a
 negative tensor power is a tensor power of the dual bundle.  The conclusion is that
 if we have a circle bundle over $Q$ whose curvature is $\w$, we can construct an
 action of $C^\infty(M)$ such that poisson brackets go to commutator brackets.  If we
 think of them as sections of a complex line bundle, what are these operators?  We
 have vector fields on the base, and a connection gives us a covariant derivative.
 You can read Kostant's article in Lec.~Notes in Math.~170.

 How do we know there is such a $Q$?  We know that we can take the trivial bundle in
 the case of a cotangent bundle (or whenever $\w$ exact).  What if $\w$ is not $d$ of
 some 1-form, as in the case of a compact symplectic manifold.
 \begin{theorem}[A. Weil]
   $\w\in \W^2(M)$ is the curvature of a $U(1)$-bundle over $M$ if and only if $[\w]\in
   Im(H^2(M,\ZZ)\to H^2(M,\RR))$ (where this is the map corresponding to coefficient
   homomorphism $\ZZ\xrightarrow{2\pi} \RR$) if and only if $\int_\sigma \w \in 2\pi
   \ZZ$ for every 2-cycle $\sigma$ on $M$.
 \end{theorem}
 This is often called the integrality condition.  In this context, it is also called
 the prequantization condition.  If we take the standard 2-sphere in $\RR^3$, with the
 usual area element, it is prequantizable.  In $\g^* = \mathfrak{su}(2)^*$, we have
 $\{x,y\}=z, \{y,z\}=x, \{z,x\}=y$, and the levels of $x^2+y^2+z^2$ are the symplectic
 leaves  ... coadjoint orbits.  Then the symplectic area of the sphere is equal to
 $4\pi r$.  The quantizable ones are those for which $r\in \frac{1}{2}\ZZ$ (including
 $r=0$).  This corresponds to spin.  The representations of $SU(2)$ are indexed by
 this spin.  But $\mathfrak{su}(2)=\mathfrak{so}(3)$, but the reps of $SO(3)$ require
 spin to be a whole integer.  In general, for complact simply-connected groups, there
 is one to one correspondence between irreducible representations and quantizable
 coadjoint orbits (Borel-Weil Theorem).  This is also true for nilpotent groups
 (Kirillov).

 $0\to \ZZ\xrightarrow{2\pi[i]} [i]\RR \to [i]\RR/2\pi[i]\ZZ \to 0$.  Some people put
 the $i$ because then the quotient is more directly isomorphic to $U(1)$.  There is a
 corresponding long exact sequence in cohomology (in the same direction (these are the
 coefficients))
 \[
    \cdots\to H^1(M,U(1)) \to \underbrace{H^2(M,\ZZ) \xrightarrow{2\pi} H^2(M,\RR)} \to H^2(M,U(1))
\to H^3(M,\ZZ)\to \cdots
 \]
 The condition for prequantizability is that $[\w]$ lies in the image of $H^2(M,\ZZ)$.
 It turns out that $H^2(M,\ZZ)$ classifies [hermitian, unitary] complex line bundles
 up to isomorphism.  The element of $H^2(M,\ZZ)$ corresponding to the bundle is called
 the chern class $c_1$.  The kernel of the underbraced map are the complex line
 bundles whose churn classes go to the symplectic class.  There is a lack of
 uniqueness if the previous map is non-trivial.  If there is a connection whose
 cohomology is zero, then there is a connection which is flat (curvature is
 identically zero).  Remember that there is a group structure on complex
 line bundles (given by tensor product), which makes the map to $H^2(M,\ZZ)$ a group
 homomorphism.  The point is that if you have two line bundles with connections
 $curv((E_1,\ phi_1)\otimes (E_2,\ phi_2)) = curv(E_1,\phi_1) + curv(E_2,\phi_2)$.  So
 there is another group here, the one of hermitian line bundles with connection.

 \[\xymatrix @!0 @C=12mm @R=15mm {
  & & & \tiny\txt{bundles with\\ connection} \ar[dl]_{\tiny\txt{forget\\ connection}} \ar[dr]^{\tiny\txt{curvature}} \\
  & & \tiny\txt{line\\ bundles} \ar[d]_{\tiny\txt{chern\\ class}} & & Z^2(M) \ar[d]^{[\ ]}\\
   \cdots \ar[rr] & & H^2(M,\ZZ)\ar[rr]^{2\pi} & & H^2(M,\RR)\ar[rr] & & \cdots
 }\]
