 \stepcounter{lecture}
 \setcounter{lecture}{10}
 \sektion{Lecture 10 - Hamiltonian Mechanics}

 Submit a proposal for a term paper by Tuesday via bspace.

 If we start with a symplectic manifold $(M,\w)$, we have
 $TM\xrightarrow{\tilde \w} T^*M$.  Given $H\in C^\infty(M)$, we
 can look at $X_H:=\tilde \w^{-1} (dH)$, which is called the
 \emph{Hamiltonian vector field} generated by $H$, or the
 ``symplectic gradient of $H$''.  This vector field has some nice
 properties:
 \[
   \L_{X_H}H = dH(X_H) = \tilde \w (X_H)(X_H)=\w(X_H,X_H)=0
 \]
 This is conservation of energy ($X_H$ is flow through time, and
 $H$ is energy).  We also have
 \[
    \L_{X_H}\w = i_{X_H}\underbrace{d\w}_0 + di_{X_H}\w = d\tilde\w(X_H) = ddH=0
 \]
 so $\w$ is invariant in time.  From this you can derive
 \[
    \L_{X_H}\w^k=0
 \]
 for any $k\ge 0$.  This give us that $\L_{X_H}\w^{\dim M/2}=0$,
 which says that hamiltonian flows preserve volume of phase space.

 $X_H=0$ at critical points of $H$, which are called equilibria.
 You cannot have stable (or completely unstable) equilibria in
 hamiltonian mechanics because stable would give volume loss, and
 unstable would give volume gain.

 Consider the case of a pendulum, with $\w=dq^i\wedge dp_i$, then
 \[
    dH = \pder{H}{q^i}dq^i+\pder{H}{p_i}dp^i
 \]
 \[
    X_H = -\pder{H}{q^i}\pder{}{p_i} + \pder{H}{p_i}\pder{}{q^i}
 \]
 so
 \begin{align*}
   \der{q^i}{t} &= \pder{H}{p_i}\\
   \der{p_i}{t} &= -\pder{H}{q^i}
 \end{align*}
 which are called Hamilton's equations.  $H=\frac{1}{2}p^2+V(q)$.
 In our case, $q$ is the angle of the pendulum, $V(q)$ is height.
 Phase space looks like this, with level curves:

 figure 1

 Lagrange was studying celestial mechanics.  He wrote down these
 equations for this.  An elliptical orbit is given by six
 parameters.  If there are multiple planets, then for each planet,
 the six parameters evolve in time.  Later Hamilton did stuff.

 Suppose we look at an arbitrary vector field $X$, then
 \[
    \L_X \w  = d(\tilde \w(X))=0\Leftrightarrow
    X=\tilde\w^{-1}(\alpha)
 \]
 where $d\alpha=0$, and $\alpha$ is determined by $X$.  The flows
 preserving $\w$ are those corresponding to closed 1-forms.  The
 hamiltonian vector fields are those  that correspond to exact
 1-forms.  These are the same locally, so such $X$ are called
 ``locally hamiltonian''.

 If we look at the cylinder, with coordinates $(\theta,z)$, with
 $\w=d\theta\wedge dz$.  Then $\pder{}{z}$ is closed, but it would
 have to be ``$-d\theta$'', so it is not globally hamiltonian.

 Define $\{f,g\} = \w(X_f,X_g)$.  So $\{q^i,p_j\} =
 \w(-\pder{}{p_i},\pder{}{q_j})=\delta_j^i$.  In general, we have
 \[
    \{f,g\} = \pder{f}{q^i}\pder{g}{p_i} -
    \pder{f}{p_i}\pder{g}{q^i}
 \]
 (note that we are summing over $i$).  We can also write
 $\{f,g\}=\pi(df,dg)$, where $\pi=\pder{}{q^i}\wedge\pder{}{p_i}$
 (again, we are summing over $i$).  This is called a
 \emph{bivector field}, and gives us a \emph{Poisson structure}.
 We need to show that the Jacobi identity is satisfied.
 Note that $\{f,g\}=X_g f$.

 The flow of $X_g$ is hamiltonian, so the flow preserves
 symplectic structure, and therefore preserves poisson brackets.
 Let $\phi_t$ be the flow of $X_h$, then we have
 \[
    \{\phi_t^*f,\phi_t^* g\} = \phi_t^*\{f,g\}
 \]
 So you have a family of automorphisms.  Suppose $B$ is a bilinear
 operation, $A_t$ is a family of automorphisms with $A_0=\id$, and you have
 $B(A_tf,A_tg)=A_tB(f,g)$.  Then differentiating with respect to
 $t$ (setting $X=\pder{A_t}{t}{\big|_{t=0}}$)
 \[
    B(Xf,g)+B(f,Xg) = XB(f,g)
 \]
 In our case, we have $B=\{\ ,\, \}$, and $A_t=\phi_t^*$, so we
 have
 \[
    \{X_hf,g\}+\{f,X_hg\} = X_h\{f,g\}
 \]
 , that is, $X_h$ is a derivation of the bracket.  This is the
 derivation property of hamiltonian vector fields.  Jacobi first
 wrote this, so it is called the Jacobi identity.  To see that
 this is the regular Jacobi identity, recall the definition of
 $X_hf = \{f,h\}$:
 \[
   \{\{f,h\},g\}+\{f,\{g,h\}\} = \{\{f,g\},h\}
 \]
 This is called the \emph{right Liebniz property} (i.e. bracketing
 on the right is a derivation).

 We also have that this is anti-symmetric, so we can put
 re-arrange everything to look like
 \[
   \{g,\{h,f\}\}+\{f,\{g,h\}\}+\{h,\{f,g\}\}=0
 \]
 which is the usual Jacobi identity.  (Note: there are some things
 that are Liebniz algebras, but are not anti-symmetric)

 Some more manipulating yeilds
 \begin{align*}
   \{f,\{g,h\}\} &= \{\{f,g\},h\} - \{\{f,h\},g\} \\
   X_{\{g,h\}} f &= X_hX_g f - X_gX_h f\\
            &= -[X_g,X_h]f
 \end{align*}

 Some examples of Liebniz algebras:
 \begin{itemize}
 \item[(a)] $M_n(\RR)\oplus \RR^n$, with the rule
 \[
    [[(A,v),(B,w)]] = ([A,B],Aw)
 \]
 which is called a \emph{hemi-semidirect product}  (the semidirect
 product would be $    [[(A,v),(B,w)]] = ([A,B],Aw-Bv)$)
 \item[(b)]  What we did before was
 \[
    [[(X,\alpha),(Y,\beta)]] = ([X,Y],\L_X \beta - \L_Y\alpha +
    \frac{1}{2}(d(X_2\lrcorner \alpha_1) - d(X_1\lrcorner
    \alpha_2)))
 \]
 then something if we use $\L_X\beta -i_Y\alpha$ in the second
 part, we get another bracket $[[[\ ,\, ]]]$.  You then have
 \[
    \frac{1}{2}([[[e_1,e_2]]] - [[[e_2,e_1]]]) = [[e_1,e_2]]
 \]
 \end{itemize}

 So far we have two properties of the Poisson bracket:
 \begin{align*}
   X_h\{f,g\} &= \{X_hf,g\} + \{f,X_h g\}\\
   X_h fg &= (X_hf)g + f(X_hg)
 \end{align*}
 we know that $\phi_t^*(fg)=(\phi_t^* f)(\phi_t^*g)$, and
 differentiating like before gives the second rule, which we can
 write as
 \[
    \{fg,h\} = \{f,h\}g + f\{g,h\}.
 \]
 What we have is two operations: pointwise product and bracket.
 One is commutative and the other is anti-commutative, and they
 are linked in this way.  When we have all this, we say we have a
 \emph{Poisson Algebra}.

 Poisson showed that if $\{f,h\}=\{g,h\}=0$, then
 $\{\{f,g\},h\}=0$.  He knew that time evolution was given by
 bracketing, so it was important to look for invariants of motion,
 and this says that if you have two invariants, their bracket is
 another one (and it is sometimes something really
 new\footnote{For example, bracketing $x$ and $y$ angular momentum
 yields $z$ angular momentum.}).

 Jacobi first introduced the idea of vector fields as operators.
 Lie's contribution was  the observation that you can study
 abstractly things that satisfy this identity.

 If you have a Hamiltonian vector field $X_h$ and a submanifold $S$, when is
 the vector field tangent to the submanifold?

 \begin{itemize}
 \item[(1)] If $S$ is a level of $h$, then yes (conservation of energy).
 \item[(2)] If $S$ is an orbit of $X_h$, then yes.
 \item[(3)] If $S$ is a level of $f$ then if and only if $\{f,h\}=0$ on $S$ if
 and only if $\{h,f\}=0$ on $S$, which is equivalent to saying
 that $h$ is a constant of motion along the flow of $f$!  This
 goes under the name of Noether's theorem.

 This says that looking for invariants of motion along the flow of
 $h$ is equivalent to looking for functions whose hamiltonian flow
 preserve $h$.  In other words, you are looking for symmetries of
 $(M,\w,h)$.  So looking for conserved quantities is equivalent to
 looking for symmetries.
 \item[(4)] If $S$ is lagrangian, then the condition is that
 $X_h\in TS = (TS)^\perp$.  This is the same as saying that
 $dh = \tilde\w (X_h)\in \tilde\w (TS)^\perp = (TS)^0$ (the forms which
 annihilate $TS$).  This is equivalent to saying that $h$ is
 constant on $S$.

 \item[(5)] In the special case where $M=T^*Q$, and $S=$ image of
 $d\phi$, where $\phi\in C^\infty(Q)$.  The image of $d\phi$ is
 invariant under the flow of $X_h$ if and only if $h\cdot d\phi$
 is constant.  That is, $h(q,\pder{\phi}{q})=$constant ... this is
 the \emph{Hamilton-Jacobi} equation.
 \end{itemize}
