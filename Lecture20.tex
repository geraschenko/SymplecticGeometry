 \stepcounter{lecture}
 \setcounter{lecture}{20}
 \sektion{Lecture 20}

 Our basic symplectic manifold is $\CC^d$.  The idea of the Delzant construction is
 that we start with a delzant polytope $\Delta$ in ${\t^n}^*$, given by $\langle
 x,v_i\rangle \le \lambda_i$ for some numbers $\lambda_i, v_i$.  We use these $v_i$s
 to construct $a:\t^* \to \RR^d$, letting the $i$-th component be $\langle
 x,v_i\rangle$.  Then $\Delta = a^{-1}(\lambda+\O_-)$, where $\O_-$ is the negative
 orthant.

 Figure 1

 \def\n{\mathfrak{n}}

 $\t^d$ acts naturally on $\CC^d$, and the momentum map is $J(z_1,\dots,
 z_n)=-\pi(|z_1|^2,\cdots ,|z_n|^n)+\lambda$.  Then the negative orthant is the image
 of the momentum map.  Let $\n\subseteq \t^d$ be the elements of $\t^d$ which
 annihilate $\n^\perp$. Then $\n^*\simeq \t^{d*}/\n^\perp$.  This corresponds to some
 $N\subseteq \TT^d$ (this follows from the Delzant condition on the polytope).  Now we
 restrict from the $\TT^d$ action on $\CC^d$ to an $N$ action.  The momentum map
 $J_N\CC^d\to \n^*=\t^{d*}/\n^\perp$ is composition of $J$ with the projection onto
 $\n^*$.  Next we form the reduced manifold $\CC^d_0$, reduced by the $N$ action.
 That is, we just take the part of $\CC^d$ which maps to $\n^\perp$ (this is compact
 because $J$ is proper).  This is $J_N^{-1}(0)$.  The reduced manifold is
 $J_N^{-1}(0)/N$.  It turns out that $J_N^{-1}(0)$ is a manifold, and that $N$ acts
 freely on it, so our reduction is a smooth manifold.  The entire torus still acts on
 this manifold (because the torus action commutes with the $N$ action).  Choose a
 complement $\m^\perp\subseteq \t^{d*}$ to $\n^\perp$, to which there corresponds some
 $M\subseteq \TT^d$.  Thus, $\TT^d=M^n\times N^{d-n}$.  Now we look at the action of
 $M$ on the reduced manifold.  Notice that $M$ is an $n$-dimensional torus, so we can
 identify $\m^*\simeq \t^{d*}/\m^\perp$, which we may identify with $\n^\perp$, which
 is identified with $\t^{n*}$.  Thus, we identify $M$ with our original $\TT^n$.  What
 is $J_M$?  There is a quotient map from $J_N^{-1}(0)$ to $J_N^{-1}(0)/N$.  The image
 of the momentum map is our original polytope $\Delta$.

 You can often answer questions about polyhedra by realizing them as the images of
 momentum mappings.

 \subsection*{Geometric Quantization}
 Souriau introduced this term.  The idea was developed over some time by VanHove (who
 directly influenced Souriau), Segal, Kirillov, Kostant.  There is a dictionary
 between classical and quantum mechanics

 \hspace{-2cm}\begin{tabular}{c|c|c} \hline
   Classical & Quantum & What it is \\ \hline
   symplectic manifold & hilbert space & ``phase space''\\ \hline
   symplectic mappings & unitary tranformations & time evolution\\ \hline
   locally hamiltonian vector fields & $i\cdot$hermitian operators & infinitesimal time evolution\\
   (functions modulo constants) & ($i\cdot$hermitian modulo scalar)\\ \hline phase
   space $T^*X$ & hilbert space $L^2(X)$\footnote{You don't actually need a measure
   ... these things are called half-measures, or half-densities.} \\ \hline
   $M$ & $Quan(M)$ & say you have such a procedure\\ \hline
   $M\times N$ & $Quan(M)\otimes Quan(N)$ & \\ \hline
   $T^*X\times T^*Y$ & $L^2(X)\widehat{\otimes} L^2(Y)$ \\
   $=T^*(X\times Y)$ & $=L^2(X\times Y)$ \\ \hline
   $C^\infty(T^*X)$ & Operators on $L^2(X)$ \\ \hline
 \end{tabular}
 Logically, you should be able to go from quantum to classical, and illogically, you
 should be able to go the other way.  There are several ways to do this.

 If $X=\RR$, and $T^*X$ has coordinates $q,p$, then $q\mapsto m_x$, multiplication by
 $x$ (we will say $m_x=Quan(q)$), and $p\mapsto i\hslash \pder{}{x}$.  What should $qp$ go to?  It should go to
 $i\hslash m_x D_x$, but $pq$ ``should'' go to $i\hslash D_x m_x$.  These are
 classically equal, but quantumly not (unless $\hslash\to 0$).  But since we were
 thinking of these operators as the lie algebra of unitary operations, so
 multiplication is not the big operation ... the lie bracket is!  So let's look at
 $m_x D_x - D_x m_x = -m_1$.  Thus, we have
 \[
    [Quan(q), Quan(p)] = -i\hslash m_1 = -i\hslash Quan(1) = -i\hslash Quan(\{q,p\})
 \]
 which is great.  It turns out that this is not exact either (this $=$ poisson bracket
 yields lie bracket), but you can get it to come out right modulo higher order terms.

 This is an infinitesimal version of saying that we'd like the group of symplectic
 mappings to go to the group of unitary transformations (as groups).  But this doesn't
 work in general.  You can make it work modulo higher order terms, or you can ask it
 to work for some subset of symplectic mappings.

 There are some nice papers written in the last few years by N.~Landsman (what should
 go on the RHS if you put the category where the morphisms are lagrangian submanifolds
 of products on the LHS?).

 If $M$ is symplectic.  How do we assign a vector space to $M$ in such a way that
 symplectic transformations correspond to unitary operators?  Try $C^\infty(M)$, or
 $L^2(M)$ with respect to the symplectic measure $\frac{\w^n}{n!}$, where $\dim M=2n$.
 These don't give you the right answer for $M=T^*\RR^n$ ... it's too big!  Another
 problem is when you try to assign operators to functions.  Say $f\in C^\infty(M)$
 gives you $i\cdot X_f$.  Then if you take $f=1$, you get the operator 0, not
 multiplication by 1, so this is no good.  You might think you should just take
 $iX_f+m_f$, but this doesn't work either.

 \underline{Polarization}: takes care of the ``too big'' problem.  Choose a Lagrangian
 foliation of $M$ ... this is called a real polarization.  Now instead of looking at
 all the functions on $M$, look only at functions constant on the leaves of the
 polarization.  If you take $T^*X$ with the polarization by fibers, then the space of
 leaves is $X$, so we are looking at functions on $X$.  How should symplectic
 transformations act on this space?  Symplectic transformation do not preserve the
 polarization in general.  So lets only look at functions preserving the fibration.
 There are two sources of such things: diffeomorphisms of $X$, and 1-forms.
 $p\rightsquigarrow i\hslash X_p = i\hslash \pder{}{q}$, $q\rightsquigarrow i\hslash
 X_q = -i\hslash \pder{}{p} = 0$ since we are acting on things independent of $p$.
 phooey.  We will use something like $i\hslash X_f + m_{g}$, but $g$ cannot just be
 $f$, it has to depend on $f$ in some more complicated way.

 \underline{Prequantization}: takes care of the problem with the constants.  Think of
 $q,p$ space.  The problem with using just the hamiltonian vector fields was that the
 vector fields commute, and $\{q,p\}\neq 0$.  So we add another dimension, which we
 call $\theta$.

 FIgure 2

 We need a distribution in $q,p,\theta$ space.  Consider the 1-form $i\hslash d\theta-pdq$.
 Now consider the horizontal lifts with respect to this form.  $q\rightsquigarrow
 -\pder{}{p} \rightsquigarrow  -\pder{}{p}$, but $p\rightsquigarrow \pder{}{q}
 \rightsquigarrow \pder{}{q} + \frac{}{i\hslash} p \pder{}{\theta}$.  Now if we take
 the bracket of these two vector fields, we don't get zero any more.  Let's look at
 functions of the form $f(q,p)\theta$.  And let's introduce a polarization by
 requiring that $f$ depends only on $p$, then $q$ goes to $\pder{}{p}$, and $p$ goes
 to something which gets rid of $\theta$.  So let's use $e^{i\theta}$ instead of
 $\theta$.  Now $p$ gives us something closer to $m_p$.
