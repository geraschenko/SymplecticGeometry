 \stepcounter{lecture}
 \setcounter{lecture}{25}
 \sektion{Lecture 25 -  Geometric Quantization to Deformation Quantization}

 Geometric quantization: $(M,\w)\rightsquigarrow \H$ so that
 $C^\infty(M)\rightsquigarrow $ operators on $\H$.  Then $\{,\}\rightsquigarrow [,]$,
 and hamilton's equations $\der{f}{t} = \{f,H\}$ become $\der{A}{t}=i\hslash [A,\hat
 H]$ for some operator $A$.  If you can get at this algebra, then you're good ... the
 $\H$ is a representation of the algebra.

 Deformation quantization: forget about $\H$ and just try to get the algebra.  Start
 with $C^\infty(M[,\CC])=A_0$, and embed it into a family of algebras $A_\hslash$.
 Let's assume that the $A_\hslash$ are all the same (as vector spaces).  So on some
 vector space, we have a family $\ast_\hslash$ of associative products.  We will
 assume that $\ast_0$ is commutative.  Then
 \[
    0=\der{}{\hslash}{\Big |}_0 \left[ (f\ast_\hslash g)\ast_\hslash h - f\ast_\hslash (g\ast_\hslash h) \right]
 \]
 which gives you a condition on $\left. \der{}{\hslash}\right|_0 (f\ast_\hslash g) =
 B_1(f,g)$.
 \[
    f \ast_\hslash g = f\ast_0 g + \hslash B_1(f,g) + O(\hslash^2)
 \]
 we will write $fg$ for $B_0(f,g)$, and assume it is commutative. We have
 \[
    \delta B_1 (f,g,h) = B_1(f,g)h + B_1(fg,h) - B_1(f,gh) - fB_1(g,h)=0
 \]
 which is part of the Hochschild complex.  Look at $B_1(f,g)-B_1(g,f)$ ... it turns
 out to be a bi-derivation, which we will call $\{,\}$. If
 \[
   f \ast_\hslash g = f\ast_0 g + \hslash B_1(f,g) + \hslash^2 B_2(f,g)+ O(\hslash^3)
 \]
 for some $B_2$, then $\{,\}$ satisfies Jacobi identity.  $\{f,g\}$ is the first order
 part of the commutator of $f$ and $g$, and gives a Poisson structure.

 If you start with a Poisson manifold, does there exist a deformation such that the
 above construction gives the given poisson bracket?  It is hard to define these
 $\ast_\hslash$ for a given $\hslash$.  Given $B_0,B_1,\dots :A\times A \to A$, where
 $B_0$ is multiplication, they define a bilinear product on $A[[\hslash]]$ given by
 \[
    a\ast_\hslash b = \sum_{j=0}^\infty B_j(a,b)
 \]
 where the $B_j$ are extended to $A[[\hslash]]$ by $\CC[[\hslash]]$-linearity.

 Today we'll talk about how on a K\"ahler manifold, you can go from a geometric
 quantization to a deformation quantization; this is sometimes called Berezin-Toeplitz
 quantization, but the people who did the first work were Boutet de Monvel and
 Guillemin.

 $(M,\w,Q,\phi,F)$ with $M$ compact, from which we get a vector bundle $E$, with
 $\Gamma(E)\simeq\H_{-1}$, and $\Gamma_{hol}(E)\simeq \H_{-1,F} \subseteq \H_{-1}$, and
 $\E^n=\Gamma_{hol}(E^{\otimes n})\simeq \H_{-n,F}\subseteq \H_{-n}$.  The Riemann-Roch theorem tells us
 about what happens as $n$ gets big.  Let $\H_{-n}\xrightarrow{\pi_n} \H_{-n,F}$ be
 the projection.  For $f\in C^\infty(M)$, define the $n$th Toeplitz operator $T_f^{(n)} = \pi_n M_f
 \pi_n: \E^n\to \E^n$.
 \begin{theorem}[Boutet de Monvel-Guillemin, Bordemann-Meinrenken-Schlichenmaier,...]
   There are (unique!) bidifferential operators $B_j:C^\infty{M}\times C^\infty{M} \to
   C^\infty{M}$ such that, if we set $f\ast_{[I]}g = \sum_{i=0}^I \frac{1}{n^i}
   B_i(f,g)$, then $||T^{(n)}_{f\ast_{[I]} g}-T^{(n)}_fT^{(n)}_g||\le
   C_{I}\frac{1}{n^{I+1}}$.  So as $I\to \infty$, $T^{(n)}$ gets closer and closer to
   a homomorphism.  In particular, $B_0(f,g)=fg$, $B_1(f,g) -B_1(g,f) = i\{f,g\}$.
 \end{theorem}
 $\{f,g\} = \pder{f}{q_i} \pder{g}{p_i} - \pder{f}{p_i} \pder{g}{q_i}$, we may write
 $P = \pder{}{q^i}\pder{}{p_i} - \pder{}{p_i}\pder{}{q^i}$, with the operators acting
 in the right direction, then we have $fPg$.

 It follows from the theorem that the $B_i$ form a deformation quantization.  The
 theorem tells you about the existence of a deformation quantization, but it also gives
 you information about the holomorphic sections.  All the proofs use the picture of a
 tower of line bundles.  The $T^{(n)}_f$ act on different spaces, so that's annoying,
 but all the $\H_{-n}\subseteq L^2(Q)$, so you can look at the direct sum of all the
 spaces in $L^2(Q)$.

 In the simplest example, $M=\PP^1=S^2$, then $Q=S^3$.  Then the $\H_{-n}$ are degree
 $n$ homogeneous polynomials, so when you take the direct sum and take the appropriate
 closure, you get holomorphic functions on the 4-disk.  You can look at those
 functions which are in $L^2$[?] on $S^3$.

 What do the operators $B_i$ look like?  In local coordinates, it's too complicated to
 write the formula.  Let's assume that there is a piece of the manifold which is flat.
 Then in local coordinates in a flat place,
 \[
    B_1(f,g) = \frac{1}{i}(\underbrace{\pder{}{z^j}_{\leftarrow}\pder{}{\bar z^j}_\to }_\Phi) g
 \]
 \begin{align*}
    \pder{f}{z}\pder{g}{\bar z} &= \frac{1}{2}(f_q-if_p)(g_q+ig_p)\\
        &= \frac{1}{2}(f_qg_q+f_pg_p+i(\underbrace{f_qg_p+f_pg_q}_{\{f,g\}}))
 \end{align*}
 We have that
 \[
    f\ast_\hslash g = f(e^{\hslash \Phi})g.
 \]
 where $\Phi^n = \partial_z^n \partial_{\bar z}^n$, or whatever.
 This is also called the [Anti-?]Wick product.  If $g$ is holomorphic or $f$ is
 anti-holomorphic, then $f\ast_\hslash g=fg$.

 This is a local model for compact manifolds, but the product also works globally on
 $\CC^n\simeq \RR^{2n}$.  Since tangent spaces of K\"ahler manifolds look like
 $\CC^n$, this is the local picture in general.  There is a real theory too, which was
 used in the more algebraic proofs of deformation quantization.  In that case, the
 local model is the Moyal Product on $\RR^{2n}$ (Von Neumann, Weyl):
 \begin{align*}
    f\ast_\hslash g  &= f(e^{\frac{i\hslash}{2}P})g\\
     &= fg + \frac{i\hslash}{2} \{f,g\} + (\frac{i\hslash}{2})^2\frac{1}{2!} f(P)^2g +
     \cdots
 \end{align*}
 this is a standard star product, and it is unique if you assume that it is invariant
 under the action of the symplectic group $Sp(2n;\RR)$, which acts on functions by
 pulling back.  On quadratic functions, $\mathfrak{sp}(2n;\RR)$, $f\ast_\hslash g -
 g\ast_\hslash f = i\hslash\{f,g\}$ (it is clear that the third and higher order terms
 vanish, but so do the second order terms!).  In fact, you only need one of $f,g$ to
 be quadratic.  This thing lives naturally on any symplectic vector space, so you have
 one on each tangent space of a symplectic manifold.

 Another possible product:
 \begin{align*}
   f\ast'_\hslash g &= f(e^{i\hslash \pder{}{q}_{\leftarrow} \pder{}{p}_\to})g
 \end{align*}
 which is much simpler.

 You want to get from Polynomials on $\RR^{2n}$ to operators on $L^2(\RR^n)$.  We take
 $q\mapsto M_z$, $p\mapsto i\hslash \pder{}{x}$.  Then where does $qp$ go?  It could
 be $i\hslash x \pder{}{x}$ or $i\hslash \pder{}{x} x$, or if you can't decide, you
 can take $\frac{i\hslash}{2}(x\pder{}{x}+\pder{}{x}x)$.  The first rule corresponds
 to the star product $\ast'_\hslash$ (that's where composition goes), whereas the last
 one leads to the Weyl product.

 In general, the Weyl product doesn't converge ... it is just formal.  But there is an
 integral formula.  Let $x,y,z$ be general points in $\RR^{2n}$.
 \[
    (f\ast_\hslash g)(z) = (something) \int K(x,y,z) f(x)g(y)\, dx\, dy
 \]
 To get the Weyl product, take $K = e^{\frac{i}{\hslash} 4Area(\Delta_{xyz})}$ where
 $\Delta_{xyz}$ is the triangle with vertices $x,y,z$.  This is how Von Neumann comes into
 the story.

 What does this have to do with the Weyl product formula we had?  Principal of
 Stationary Phase: if $S$ a morse function,  $\int e^{\frac{i}{\hslash} S(x)}a(x)\, dx
 \sim_{\hslash\to 0} \sum_{p\in crit\, S} \hslash a(p)\frac{1}{\sqrt{\det
 \frac{\partial^2 S}{\partial x^i \partial x^j}}}+\cdots$
