 \stepcounter{lecture}
 \setcounter{lecture}{5}
 \sektion{Lecture 5}

 \texttt{www.math.ist.utl.pt/\~{}acannas/Books/symplectic.pdf}
 [or ps] has corrected references.  Also, there is a nice article
 on the arXiv at \texttt{---.math.SG/0505366}.

 \[\frac{d}{dt}(f^*_t \w_t) = f_t^* \frac{d\w_t}{dt}+
 \left(\frac{d}{dt}(f_t^*)\right)\w_t\]  You can think of $f_t^*$ as a matrix
 and $\w_t$ as a vector.  And $\frac{df_t}{dt} = v_t\circ f_t$
 (these are vectors along the paths $f_t(p)$). If you have a family
 of maps $f_t:M\to N$.  It turns out that you can write
\begin{align*}
    \left(\frac{d}{dt}(f_t^*)\right)\w_t &= f_t^*\L_{v_t} \w_t
\end{align*}
 so
 \begin{equation*}
    \frac{d}{dt}(f^*_t \w_t) = f^*_t\left(\frac{d\w_t}{dt} + i_{v_t}
    d\w_t + d(i_{v_t}\w_t)\right) \tag{$\ast$}
 \end{equation*}
 by the Cartan magic formula.

  \[\begin{pspicture}(-.5,-.5)(7.8,2.5)
  \psccurve(0,0)(0,2)(1.9,1.9)(2,0) \rput(2.2,2.2){$M$}
  \psdots(.9,.8) \rput(.65,.7){$p$}
  \psccurve(5,0)(5.1,1.9)(7.5,2.1)(7.1,0) \rput(4.8,2.2){$N$}
  \pscurve{->}(2.5,1.3)(3.5,1.5)(4.5,1.3) \rput(3.5,1.8){$f_t$}
  \pscurve(5.4,.1)(5.7,1)(6.5,1.2)(7.2,1.8)
  \psdots(5.7,1) \rput(6.2,.7){$f_t(p)$}
  \psline{->}(5.7,1)(6.3,1.5)
  \end{pspicture}
  \]

 If $\xi$ is a vector field along $f:M\to N$, then we have an
 interior product/pullback operator $i_\xi:\W^\cdot (N)\to
 \W^{\cdot-1}(M)$ given by $i_\xi\w(p)(v_1,\dots, v_{k-1}) =
 (\xi(p),(Tf)(v_1),\dots)$.

 If $f_t:M\to N$ a family of maps, then [[box this]]
 \[
    \frac{d}{dt}f_t^*\w_t = f^*_t\frac{d\w_t}{dt} +
    d(i_{\frac{df_t}{dt}}\w_t) + i_{\frac{df_t}{dt}}d\w_t
 \]
 Meditate on this for a while.

 This idea was used by Moser, and then used all over the place.
 Moser used this in the case where $M$ is compact, and $\w_t$ is a
 family of symplectic forms such that $[\w_t]\in H^2(M,\RR)$ is
 constant.  Then $(M,\w_t)$ are all symplectomorphic.  Trying to
 solve the equation $g_t^*\w_0=\w_t$, it's a mess, but $f^*_t\w_t
 = \w_0$ is nice.  To solve, let $f_0=\id$ and $0=(\ast)$, so
 \[
    \frac{d\w_t}{dt} + d(i_{v_t}\w_t) = 0
 \]
 is what you want (Moser's equation).  Solve for $\w_t$'s and then
 find the $f_t$'s.  The compactness of $M$ is essential for the
 second step (you have to integrate a time-dependent vector
 field).  To solve Moser's equation
 \[
    d(\tilde\w_t (v_t)) = -\frac{d\w_t}{dt}
 \]
 ($\tilde\w_t$ isomorphism), you only have to solve an equation of
 the form
 \[
    d\alpha_t = -\frac{d\w_t}{dt}
 \]
 where $\alpha_t$ are 1-forms.  Then we have
\[
    d\left(\frac{d\w_t}{dt}\right) = \underbrace{\frac{d}{dt}(d\w_t)}_0
\]

 Consider the 2-sphere, and let $\w_t=e^t\w_0$.  Then the total
 area of the sphere is changing, so you cannot get them to pull
 back to each other.  The condition that the cohomology class is
 constant forces the volume to remain the same.  $\w_{t+h}-\w_t =
 d\theta_h$, so when you divide by $h$ and let $h\to 0$, then we
 should get $\frac{d\w_t}{dt} = d(\lim \theta_h/h)$, but it is not
 clear that $\lim \theta_h/h$ behaves.  We can take care of this
 another way:
 \[
    \int_C\frac{d\w_t}{dt} = \frac{d}{dt} \int_C \w_t = 0
 \]


 Let $Z^k(M)$ be the collection of closed $k$-forms on $M$.  You
 have to show that the exact forms are a closed sub-blah.

  \[\begin{pspicture}(0,0)(5.9,6.1)
  \psdots(1,1)(1,4.4)(5,4.4)
  \pspolygon[linearc=.1](0,3.4)(0,5.4)(2,5.4)(2,3.4) \rput(1,5.8){$Z^k(M)$}
  \psline(0,4.4)(2,4.4) \rput(1.6,4.25){\tiny exact}
  \psline{->}(.3,5)(1.2,5) \rput(.7,5.15){\tiny path}
  \psline{->}(2.5,4.4)(4.5,4.4)
  \psline(5,3.4)(5,5.4) \rput(5,5.7){$H^k(M;\RR)$} \rput(5.25,4.4){$0$}
  \pspolygon[linearc=.1](0,0)(0,2)(2,2)(2,0) \rput(3,.8){$\W^{k-1}(M)$}
  \psline{->}(1,2.2)(1,3.2) \rput(.8,2.7){$d$}
  \psline(1,0)(1,2) \rput(.6,1.7){\tiny closed}
  \rput(4.1,2.3){$\stack{\text{\small complement to}}{\text{\small closed forms}}$}
  \psline(0,1)(2,1) \psline{->}(2.9,2)(1.7,1.1)
  \end{pspicture}
  \]

 We've solved for a particular $\alpha_t$, now we have to make it
 smooth with respect to $t$.

 Suppose we know that $\w_t=\w_0+t(\w_1-\w_0)$ are symplectic.
 Then $\frac{d\w_t}{dt}=\w_1-\w_0$, so we only have to solve once.
 But in going on this straight line, we might stop being
 symplectic (the condition of being non-degenerate may fail).
 Since the collection of non-degenerate 2-forms is a dense open subset
 of all forms, so you can take a tubular neighborhood of some
 path, then take a bunch of straight steps staying inside of the
 space of symplectic forms, always staying within the
 neighborhood.  This shows that if there is some path that works,
 it can be replaced by a piece-wise linear path.

 Thus, you cannot deform symplectic structure within the same
 cohomology class and get something new.  If we are trying to
 classify symplectic structures, then we have a continuous
 invariant: cohomology class.  If you only move a little in
 cohomology, then you stay symplectic [[why?]], so symplectic
 structure is not too rigid.

 \underline{non-compact case}:  Take the plane with a disk.  Take
 a cylinder, and a half cylinder, which have the same area, but
 are not symplectomorphic. \emph{Greene-Wu}: given two symplectic
 structures with the same orientation and the same total volume,
 they are equivalent (by an end-preserving diffeomorphism) if and
 only if either
 \begin{itemize}
 \item[-] both have the same finite area, or
 \item[-] both have infinite area, and both are finite on the same
 set of ends.
 \end{itemize}

 The proof is a version of Moser's method.  The cohomology part is
 trivial (a non-compact 2-d manifold has $H^2=0$), but you have to
 worry about running off your manifold when you solve for $v_t$.

 In 4 dimensions or higher, consider $(q_1,q_2,p_1,p_2)$

 \vspace{27mm}
 \[\hspace{-45mm}
   \psellipse(2,1.5)(1.7,1.2)
   \psline[linecolor=gray]{->}(2,0)(2,3.5) \rput(2.8,3.3){(q_2,p_2)}
   \psline[linecolor=gray]{->}(0,1.5)(4.5,1.5) \rput(5.2,1.5){(q_1,p_1)}
   \psline[linecolor=darkgray,linewidth=1.5pt]{->}(2,1.5)(2,2.7)
   \rput(1.75,2.1){a_2}
   \psline[linecolor=darkgray,linewidth=1.5pt]{->}(2,1.5)(3.7,1.5)
   \rput(2.7,1.25){a_1}
 \]

 The volume of the ellipse is proportional to $a_1^2a_2^2$.  If
 you change the volume, you change the symplectic structure.  If
 you change the $a_i$'s around so as to preserve volume, can you
 retain the symplectic structure?  Or if you take a rectangle of
 the same volume?  \emph{Gromov}: $\max(a_1,a_2)$ is a symplectic
 invariant, and therefore, so is the minimum.  The argument is
 based on the following.  Take a much larger symplectic manifold
 (cylinder)

 Then if $\max(b_1,b_2)> a_1$ (from some other ellipsoid), there
 is no symplectic embedding of $E(b_1,b_2)$ into $E(a_1,\infty)$
 (the cylinder).  This is Gromov's \emph{non-squeezing theorem},
 proven with psuedo-holomorphic curves stuff. This stuff goes
 under the general notion of ``symplectic capacity''.

 Question: are the $a_i$s invariants of $E(a_1,\dots, a_n)$ (in
 dimension 2$n$)?

 It turns out that the same method Moser used can be used to prove
 local things (in particular, the Darboux theorem).  For the
 moment, consider two symplectic manifolds, $(M_1,\w_1)$ and
 $(M_2,\w_2)$ with submanifolds $N_1,N_2$, respectively.

 If $f:N_1\to N_2$, then are there neighborhoods so that $f$
 extends to a symplectomorphism.  In the case where $N_2=pt$, we
 have the Darboux theorem, and in the case $N_2=M_2$, we have
 Moser's theorem.

 If $i_j:N_j\to M_j$ are the inclusions, then a necessary
 condition is that (1)$f^*(i_2^*\w_2)=i^*_1\w_1$.  $g^*\w_2=\w_1$.
 If $g$ extends, then $g\circ i_1=i_2\circ f$.  And another
 necessary condition if that (2)$\dim M_1=\dim M_2$.

 \begin{theorem}[Givental]
   This is sufficient locally in $N$.
 \end{theorem}

 That is, the pullback form is the only local invariant.  This is
 not the case in Riemannian geometry ... you have have isometric
 morphisms of submanifolds which do not extend to isometric
 maps on neighborhoods.

 Sufficient condition to get it for all of $N_1,N_2$:
 \[\xymatrix{
 TN_1 \ar[r]^{Tf} \ar@{^(->}[d] & TN_2 \ar@{^(->}[d]\\
 T_{N_1}M_1 \ar@{^(->}[d] & T_{N_2}M_2 \ar@{^(->}[d]\\
 TM_1 & TM_2
 }\]
 The condition is that $Tf$ extends to $T_{N_1}M_1$ as a morphism
 of symplectic vector bundles.  In the case where the $N$s are
 points, it says that you have to find symplectic isomorphism
 between the tangent spaces at the points, which you can always
 do.  In fact, if $N_1$ and $N_2$ are lagrangian, then you can
 always do this.  But in any symplectic manifold $M$, if you have
 a lagrangian $N$, you can look at the cotangent bundle of $N$
 with the zero section.  Since $M$ locally looks like $\bar
 N\times N$, we win.
