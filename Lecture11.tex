 \stepcounter{lecture}
 \setcounter{lecture}{11}
 \sektion{Lecture 11}

 The Lagrangian $L:TM\to \RR$ is written as $L(x,v)$, where $x$
 are local coordinates on $M$ and $v=\der{x}{t}$.  Then given a
 path $\gamma:[a,b]\to M$, we define
 \[
    A(\gamma) = \int_a^b L(\dot\gamma(t))\, dt
 \]
 Classically, we have $L=\frac{1}{2}\dot{x}^2 - V(x)$.

 Consider paths from $r$ to $s$.  This forms a big space, and we
 look for paths which extremize (minimize or maximize) $L$.  How
 do you find such a thing?  Look at tangent vectors in path space.
 A path in path space is just a function of two variables.
 \begin{align*}
    A(\gamma_s) &=  \int_a^b L(\pder{}{t}\gamma_s(t)) \\
    \der{}{s}A(\gamma_s) &= \int_a^b
    (stuff)\pder{\gamma_s}{t}(t)\, dt
 \end{align*}
 The second thing must be zero for all variations, which implies
 that $stuff$ must be zero.  When you write it out,
 \[
    stuff = \pder{L}{x}-\der{}{t}\pder{L}{\dot{x}}=0
 \]
 This gives a system of partial differential equations called the
 \emph{Euler-Lagrange} equations.  In the classical case, we
 get
 \[
    0 = -\pder{V}{x} - \der{}{t}(m\dot{x}) = -\pder{V}{x}-m\ddot{x}
 \]

 What do these look like for more general Lagrangians?  Well we
 could have written these last equations as
 \begin{align*}
   \der{x}{t} &= \dot{x}\\
   \der{\dot{x}}{t} &= -\pder{V}{x}
 \end{align*}
 which is a vector field on $TQ$, given by
 \[
    \pder{}{t} = \dot{x}\pder{}{x} -\pder{V}{x}\pder{}{\dot{x}}.
 \]

 In general, we get
 \[
    \pder{L}{x^i} = \der{}{t}\pder{L}{\dot{x}^i} =
    \frac{\partial^2L}{\partial x^j\partial \dot{x}^i} \dot{x}^j +
    \frac{\partial^2L}{\partial \dot{x}^j\partial \dot{x}^i} \ddot{x}^j
 \]
 It would be handy to be able to invert the matrix $\left(\frac{\partial^2L}{\partial \dot{x}^j\partial
 \dot{x}^i}\right)$.  This is the \emph{Legendre condition}.  If the
 condition holds, then (E-L) are 2nd order ODEs (i.e., a vector
 field on $TQ$).



 If $L = \sqrt{\sum (\dot{x}^i)^2}$, you find that the E-L equations
 are degenerate, so the solution is not unique.  This is because
 changing the speed of a length-minimizing path leaves it
 length-minimizing.

 If you do the calculation for something of the form
 $L(x,\dot{x})=g_{ij}(x)\dot{x}^i\dot{x}^j$, it turns out that the critical
 points of this are geodesics, parameterized by arc length.

 If you let $p_i = \pder{L}{\dot{x}^i}$, then you can define $\E =
 p_i\dot{x}^i - L$, which we call ``energy''.
 \begin{theorem}
   The Legendre condition holds if and only if $\L:(x,\dot{x})\mapsto
   (x,p)$ is a local diffeomorphism if and only if
   $\L^*(\underbrace{\sum dx^i\wedge dp_i}_\w)$ is nondegenerate.
   Moreover, Lagrange's equations give hamiltonian vector field of
   $\E$ with respect to $\L^*\w$, i.e.
   $\widetilde{(\L^*\w)}^{-1}(d\E)$.
 \end{theorem}
   If the Legendre condition holds, then
   $\left(\pder{p_i}{\dot{x}^j}\right)$ is invertible, so $\dot{x}\to p$
   locally a diffeomorphism, so $(x,\dot{x})\mapsto (x,p)$ is a local
   diffeomorphism, given by
   \[\left(\begin{array}{cc}
   I & \ast\\
   0 & \frac{\partial^2L}{\partial \dot{x}^j \partial \dot{x}^i}
   \end{array}\right)\]

 Figure 1

 $dL_{(x,v)}:T_{(x,v)}TQ\to \RR$, but we can restrict it to
 $T_{(x,v)}T_xQ$, so $dL_{(x,v)}|_{T_{(x,v)}T_xQ} \in
 T_{(x,v)}(T_xQ)^* \cong (T_xQ)^*$, which is $T^*_xQ$.

 Thus, if $\pi:TQ\to Q$,
 \[\xymatrix{
 dL|_{\ker T\pi}:& TQ\ar[r]^{\L}\ar[dr]_\pi & T^*Q\ar[d]\\
 & & Q
 }\]

 The Legendre condition says that $T\L$ is invertible, which is
 equivalent to $\L$ being a local diffeomorphism, which is
 equivalent to $\L^*\w$ being symplectic.

 $\E(x,v)=\L(x,v)(v)-L(x,v)$, which is sometimes written $\E =
 \langle p,v\rangle - L$.  Thus, from any Lagrangian, you get a
 pullback 2-form.  When the Legendre condition fails, this 2-form
 is degenerate, so you cannot get a hamiltonian vector field from
 a function.

 There is another way of looking at this.  If we impose the
 \emph{strong Legendre condition}, that $\L$ is globally a
 diffeomorphism, then we define $H=\E\circ \L^{-1}:T^*Q\to \RR$.
 This is called the hamiltonian associated to this lagrangian.
 Then Lagrange's equations on $TQ$ are transfromed by $\L$ into
 Hamilton's equations on $T^*Q$ with canonical symplectic
 structure, and $H(x,p)=\langle p,\L^{-1}(x,p)\rangle -
 L(\L^{-1}(x,p))$.

 In the case $L=\frac{1}{2}m(\dot{x}^i)^2 - V(x)$, with $i=1,2,3$. Then
 $p_i=m\dot{x}^i$\footnote{If we write $\delta_{ij}\dot{x}^i\dot{x}^j$ instead of
 $(\dot{x}^i)^2$, we would get the right upper/lower indices.}  Then we
 get
 \begin{align*}
    \E &= \sum m(\dot{x}^i)^2 - \frac{1}{2}\sum m(\dot{x}^i)^2 + V(x) \\
     &= \frac{1}{2}\sum m(\dot{x}^i)^2 + V(x)\\
    H &= \frac{1}{2}\sum \frac{p_i^2}{m} + V(x)
 \end{align*}

 If you strengthen the Legendre condition by requiring that the
 matrix is positive definite, then you know that you have an
 extreme point (not just a critical point)[is this right?].  If
 you follow the solution to the E-L equations too far, you may
 reach a point where the path is not even locally minimizing (e.g.
 consider shortest length paths on a sphere).

 If we start with $H;T^*Q\to \RR$, consider $dH|_{\text{\tiny
 fibres of }T^*Q}:T^*Q\xleftarrow{\mathcal{M}=weird} TQ$.  So
 we're setting $v^i = \pder{H}{p^i}$, then define
 $L=(p^i\pder{H}{p^i}-H)\circ \mathcal{M}^{-1}:TQ\to \RR$.  If we
 take the Legendre transform of this $L$, we get $H$.

 \underline{Hamilton's principle}: Given some $H(x,p)$, define
 $L(x,p,\dot{x},\dot{p})= p\dot{x}-H$.  If we have a path in $T^*Q$, we can
 evaluate $L(\dot\gamma)=\alpha(\dot\gamma)-H\circ (projection)$.  Then
 Lagrange's equations say
 \begin{align*}
   \pder{L}{x} &= \der{}{t} \pder{L}{\dot{x}}\\
   \pder{L}{p} &= \der{}{t} \pder{L}{\dot{p}}
 \end{align*}
 so
 \begin{align*}
   -\pder{H}{x^i} &= \der{}{t} p_i\\
   \dot{x}^i-\pder{H}{p^i} &= \der{}{t}(0)=0.
 \end{align*}
 Notice that the Legendre condition fails miserably, so Lagrange's
 equations are first order equations. The solutions to this set of
 equations are just the extrema for our variational problem.
 Consider things that solve the second equation, so $\dot{x}^i=\der{x^i}{t}
 = \pder{H}{p^i}$.  Then Hamilton's lagrangian is $p_i\dot{x}^i-H$.  We
 identify solutions to this second half with paths in $Q$, not in
 $T^*Q$.
