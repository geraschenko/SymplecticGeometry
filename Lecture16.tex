 \stepcounter{lecture}
 \setcounter{lecture}{16}
 \sektion{Lecture 16 - About the Momentum Map}

 From last time:

 General reduction: $(M,\w,G,J)$, with $M_\mu=J^{-1}(\mu)/G_\mu$.

 Cotangent reduction: $M=T^*X$, $G$ acts on $X$.  $J:M\to \g^*$ linear on fibres.
 Then $J^{-1}(0)/G\simeq T^*(X/G)$, and $J^{-1}(a)/G$ could be identified with
 $T^*(X/G)$, but not canonically.

 Then we did an example where $M=\RR^4$, and we found that $J^{-1}(\mu)\simeq S^3$ and
 $M_\mu\simeq S^2$.  We had the Hopf fibration.

 Now for a real cotangent reduction example:  Let
 \[\xymatrix @C=3mm {
 S^1 \ar@{}[r]|{=} & U(1)\\
 S^3 \ar@{}[r]|{\subseteq} & \CC^2\\
 S^2 \ar@{}[r]|{\simeq} & \CC\PP^1
 }\]
 $T^*S^2\xrightarrow{H}\RR$ given by
 $H(x,\xi)=\frac{1}{2}||\xi||^2$.  Hamilton's equations for such a hamiltonian are
 exactly the equations for geodesics.  Then look at $H^{-1}(1/2)$.

 [[If $M\xrightarrow{J} \g^*$ and $H:M\to \RR$ is $G$-invariant, then $H|_{J^{-1}(\mu)}$
 induces $H_\mu$ on $M_\mu$.  $M\supseteq J^{-1}(\mu) \to M_\mu$.]]

 We have a momentum map $T^*S^3\xrightarrow{J} \RR$ given by $J(v)=\langle
 v,X_H\rangle$.  [[In the case of a surface of revolution, we have Clariout's Theorem,
 which tells us that the angle of a geodesic with the direction of rotation is
 fixed.]]  $J^{-1}(0)=$ vectors perpendicular to the Hopf vector field.

 Figure 1

 Notice that the sub-bundle of spaces perpendicular to the hopf vector field is not
 integrable.  What is the induced hamiltonian downstairs?  $H_0:T^*S^2\to \RR$ is also
 $\frac{1}{2}||\ ||^2$.   What if we replace 0 by $\mu$?  Because we have a metric, we
 may as well lift a vector to $\mu$ times the hopf vector field plus stuff.  Then we
 get the hamiltonian $H_\mu(v)=\frac{1}{2}||v ||^2+\mu^2$.  What about the flow?

 Consider the case where we have the hopf field itself, then it will project to the
 trivial flow on $S^2$.  If it is close to the hopf field, then it must project to
 small circles for the trajectories.  This suggests that if something is moving on the
 two sphere, it feels some force perpendicular to its direction of motion.  How do we
 see that?  The symplectic form on $T^*S^2$ has been modified ... it turns out to be
 modified by $\mu$ times the area form.

 Say $X$ is a Riemannian manifold with a free circle action, so $X\to X/S^1$.  Suppose
 the action preserves the metric.  If we pick a $\mu$, we get induced motion on
 $T^*(X/S^1)$.  We have $B\in \W^2(X/S^1)$ closed, which takes velocities to forces
 ... it behaves like a magnetic field.  In many cases, you can turn this around.  Say
 $X/S=Y$, and you want to describe motion in a magnetic field on $Y$.  Then we can
 describe it as motion on something one dimension higher so long as we can realize $B$
 as the curvature of some connection.  That is the case if and only if the cohomology
 class of $B$ is ``integral'' (i.e., if you integrate $B$ over any circle in $Y$, you
 get an integer).  One can also lift the potential, if there is one.  This is called
 Kaluza-Klien theory.  This stuff came up when people were trying to unite magnetic
 and gravitational forces ... the idea was to work in a 5-dimensional space rather
 than a 4-dimensional space-time.  This was the beginning of gauge theory.

 Now say we have $(M,\w,G)$, and say we have a momentum map which is not equivariant.
 So
 \[\xymatrix{
 v\in \g\ar[d]^\J \ar[r] & v_M \ar@{}[d]|{\parallel}\\
 \J(v) \ar[r] & x_{\J(v)}
 }\]
 We can ask that $\J:\g\to C^\infty(M)$ be $G$-equivariant, which would be a regular momentum map.
 \begin{align*}
   [Ad_g v]_M &= \der{}{t}{\Big |}_{t=0} (\exp(Ad_g v))\\
    &= \der{}{t}{\Big |}_{t=0} (g_M \exp(tv)_M g_M^{-1})\\
    &= (Tg_M) v_M g_M^{-1}
 \end{align*}

 \begin{align*}
   d\J(Ad_g v) &= (Ad_gv)_M\lrcorner \w\\
    &= \underbrace{g_M^{-1*}v_M} \lrcorner \w\\
    &= (g_M^{-1})^*(v_M\lrcorner \w)\\
    &= (g_M^{-1})^* (d\J(v))
 \end{align*}

 \[
    d(d(Ad_{g} v) - g_M^{-1*}\J(v))=0
 \]
 Assume $M$ connected, then $Ad_{g} v) - g_M^{-1*}\J(v)$ is some constant $\langle
 c(g),v\rangle$.  So $c(g)\in \g^*$.  So $c:G\to \g^*$.  This $c$ is zero if and only
 if $\J$ is equivariant if and only if $J:M\to \g^*$ is equivariant.

 This $c$ is a 1-cocycle on $G$ with values in $\g^*$ with the Adjoint representation:
 \[
    c(gh) \text{ ``='' } c(g)+Ad^*_{g^{-1}}c(h)
 \]
 Why are these called cocycles?  Because we have a whole complex
 \[
    C^0(G,\g^*) \xrightarrow{\delta} C^1(G,\g^*) \xrightarrow{\delta} C^2(G,\g^*)
    \xrightarrow{\delta} \cdots
 \]
 such that $\delta^2=0$, where $\delta(c)(g,h)= c(gh)-c(g)-Ad^*_{g^{-1}}c(h)$, and
 $C^i(G,\g^*)$ are maps from $G^i$ to $\g^*$.  And for $\alpha \in C^0$, we have
 $\delta(\alpha)= Ad_{g^{-1}}^*\alpha - \alpha$.

 Recall that the condition on the momentum map is $d\J(v)=v_M\lrcorner \w$, so we can
 change $\J$ by a constant, and when we do so, the $c$ varies over the cohomology
 class.

 If $p(g,h)$ is a 2-cochain, then we define
 \[
    \delta p(g_0,g_1,g_2) = - p(g_0g_1,g_2) + p(g_0,g_1g_2) +
    Ad^*_{g_0^{-1}}p(g_1,g_2) - p(g_0,g_1)
 \]


 There is another kind of equivariance.  Given $c\in Z^1(G,\g^*)$, we can define a new
 affine action of $G$ on $\g^*$ by the rule $g\cdot \alpha = Ad^{-1}_{g^*}\alpha +
 c(g)$.  This it is easy to see that $J$ is now equivariant with respect to this
 action.

 \underline{Example}: $M=\RR^2$ with coordinates $(q,p)$, $\w=dq\wedge dp$.  Let $G=\RR^2$ with
 coordinates $(a,b)$, then $\g\cong \RR^2$ with coordinates $(x,y)$.  Then we have the
 action $(a,b)_M(p,q)=(p+a,q+b)$.  What is $(x,y)_M$?  It is
 $x\pder{}{q}+y\pder{}{p}$.  Then $(x,y)_M\lrcorner \w = xdp-ydq = d(xp-yq)$ where $d$
 is the differential in $(q,p)$ space.  This is equal to $d\langle
 (p,-q),(x,y)\rangle$.  So we can say $\J(x,y)=xp=yq$.  Then $J(q,p)=(p,-q)$.  This is
 not equivariant because the coadjoint action is trivial.  The cocycle is
 $c(a,b,q,p)(x,y)=\J(x,y)(q,p)-\J(x,y)(q+a,p+b) = xb-ya$.  Then $c(a,b)=(b,-a)$.

 Instead of just the lie algebra generators $e_x$ and $e_y$, we add $e_t$ such that $[e_x,e_y]=e_t$
 and $[e_t,e_x]=[e_t,e_y]=0$.  The dual of this lie algebra has $\xi,\eta$ dual to
 $e_x,e_y$, and $\tau$ dual to $t$.  Define the lie algebra action by letting $e_t$
 act trivially: $(x,y,t)_M = x\pder{}{q}+y\pder{}{p}$.  If we take $J(q,p)=(p,-q,1)$,
 then it turns out that this action is equivariant.
