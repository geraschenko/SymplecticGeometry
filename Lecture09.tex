 \stepcounter{lecture}
 \setcounter{lecture}{9}
 \sektion{Lecture 9 - Compatibility of Symplectic \\ and Almost Complex Structures}

 Correction from last time:  We said $\W^K = \bigoplus_{p+q=k}
 \W^{p,q}$, then $d\W^{p,q} \subseteq \W^{p+1,q}+\W^{p,q+1}$.
 This used the fact that $d\w = \partial \w + \bar\partial \w$.
 It is only true when the complex structure is integrable.  In
 local coordinates $\w = \sum a_{IJ} \theta^I\wedge \theta^J$.
 When we took $d$ of this thing, we forgot to take $d$ of the
 $\theta^i$'s.

 Let $\theta\in \W^{1,0}$, then to prove that $d\theta\in
 \W^{2,0}+\W^{1,1}$, it suffices to show that $(d\theta)(x,y)=0$
 if $x$ and $y$ have type $0,1$.  We can calculate this using
 \begin{align*}
   d\theta(x,y) &= \underbrace{x\theta(y)}_0 - \underbrace{y\theta(x)}_0 - \theta([x,y])
 \end{align*}
 and the last term is zero if $T^{0,1}$ is involutive.

 Once you have that, you can deduce that
 $\partial^2=\bar\partial^2=0$.

 There is a real analog of this.  Suppose that $TM=E_1\oplus E_2$,
 then we get that $T^*M\simeq E^*_1\oplus E^*_2$.  Then we have
 \[
    \W^kM = \bigoplus_{p+q=k} \Gamma(\wedge^p
    E^*_1)\otimes_{C^\infty M}
    \Gamma(\wedge^q E^*_2)
 \]
 you get this structure on a ``bifoliated'' manifold (if both
 $E_1$ and $E_2$ are involutive).

 \vspace{5mm}

 \[\xymatrix{
  V \ar[r]^J \ar[rd]_{\tilde \w} & V \ar[d]^{\tilde g}\\
  & V
 }\]

 One way to state compatibility is to say that
 \begin{equation*}
 \w(Jx,Jy)=\w(x,y) \tag{$\ast$}
 \end{equation*}
 We define $g(x,y)=\tilde g(x)(y) = \tilde \w(J^{-1}x,y) =
 -\w(Jx,y)$.  $(\ast)$ implies that $g$ is symmetric such that
 $g(Jx,Jy)=g(x,y)$.  Any two of these define the third.  This
 triple is called a \emph{pseudohermitian structure}.  It is
 called \emph{hermitian} if $g$ is positive definite.

 Consider $\RR^2$, with coordinates $(p,q)$, and
 $J(\pder{}{q})=\pder{}{p}$.  Let $z=q+ip$, and $\w=dq\wedge dp$.
 Then we have that
 \[
   g\left(\pder{}{q},\pder{}{q}\right) = -\w\left(J\pder{}{q},\pder{}{q}\right) = 1
 \]
 Good, so we have the correct sign.

 If we are on a symplectic [almost complex] manifold, can we find
 compatible almost complex [almost symplectic\footnote{Not
 necessarily closed.}] structure.  The answers are yes.

 Consider the anulus in $\CC$, with the inner and outer circles
 glued radially.  This is a complex manifold, topologically $S^1\times
 S^1$, which has a symplectic structure.

 Figure 1

 Now if we do the same thing in $\CC^n$, we get a $S^{2n-1}\times
 S^1$, which cannot have a symplectic structure.

 If $J$ is integrable, we get a K\"ahler structure.

 Locally, every complex structure looks like $\CC^n$, and every
 symplectic structure looks like $\RR^{2n}$.  But when we put the
 two together, the $g$ has weird twists and doesn't have to be
 equivalent to the usual metric.

 What can we say anything about the geodesic flows of such things.
 I dunno.  It can be tied to the behavior of the laplacian (on
 functions), but I've never seen anything about this.

 If we think of almost complex structure as

 Figure 2

 then for $\w \in \wedge^2 V^*$, we can complexify it to
 $\w_\CC\in \wedge^2 V^*_\CC$ by requiring it to be complex
 bilinear.

 If we apply $\w_\CC$ to $V^{0,1}$, since $V^{0,1}$ is the graph
 of $J$, we have
 \begin{align*}
    \w_\CC(x+iJx,v+iJv) &= \underbrace{\w(x,v)-\w(Jx,Jv)} +
    i(\underbrace{\w(x,Jv)+\w(Jx,v)})\\
    &=0 \quad \text{if{f} $J,\w$ compatible}
 \end{align*}

 So another way to say $J,\w$ compatible is to say that $V^{0,1}$
 is lagrangian in $V_\CC$.  So looking for compatible almost
 complex structure on a symplectic manifold is the same as looking
 for lagrangian somethings, and on a K\"ahler manifold, it is the
 same as looking for lagrangian bi-foliations of the complexification.

 What about the metric?  We do a little computation.  Remember
 that under the projection from $V^{1,0}$ to $V$, $V$ gets the
 structure of a complex vector space.
 \begin{align*}
   \w_\CC(\underbrace{v-iJv}_{1,0},\underbrace{{w+iJw}}_{0,1}) &=
   \w(v,w)+\w(Jv,Jw) + i(\w(v,Jw) - \w(Jv,w))\\
   &= 2\w(v,w) + 2i g(v,w)
 \end{align*}
 So by taking the complexified form and restricting it to
 $V^{1,0}$, we get a hermitian form $\frac{1}{2i}\w_\CC = g -
 i\w$.  If we set $v=w$, we just get the $g$ part.

 \vspace{2mm}

 On $V\oplus V^*$, we have two natural bilinear forms:
 $\w_{\pm}((x,\alpha),(y,\beta)) = \alpha(y)\pm \beta(x)$.  If we
 have a form $\w\in \wedge^2 V$, then the graph of $\tilde \w$ is
 Dirac (is killed by $\w_+$).  On a manifold, we have $TM\oplus T^*M$.  Suppose we
 are given $\w\in \W^2M$, then the graph of $\tilde \w$ is a
 Dirac structure on the bundle $TM$ (i.e., is lagrangian
 sub-bundle of $TM\oplus T^*M$).

 T.~Courant: Bracket on section of $TM\oplus T^*M$ defined by
 \[
    [[(X,\alpha),(Y,\beta)]] = ([X,Y],\L_X\beta-\L_Y\alpha)
 \]
 If we stop here, this is a semi-direct product of lie algebras
 (because $[\L_X,\L_Y] = \L_{[X,Y]}$).  But it doesn't have the
 property that something is closed under it if and only if it is
 the graph of a closed form.  So we change it to
 \[
    [[(X,\alpha),(Y,\beta)]] = ([X,Y],\L_X\beta-\L_Y\alpha+\frac{1}{2}d(i_Y\alpha-i_X\beta))
 \]
 This is no longer a lie algebra (doesn't satisfy Jacobi) and is not a semi-direct
 product. It has the property that the graph of $\tilde \w$ is
 closed under this bracket if and only if $d\w=0$.  A lagrangian
 sub-bundle of $(TM\oplus T^*M,\w_+)$ is called an \emph{almost
 dirac} structure, and an almost dirac structure whose sections
 are closed under bracket is called a \emph{dirac structure}.
 \begin{theorem}
   If $E$ is a Dirac structure, then the restriction to $E$ of the
   bracket satisfies the Jacobi identity.  That is, $E$ is a lie
   algebra under $[[\ ,\, ]]$.
 \end{theorem}

 \[\xymatrix{
  E \ar@{^(->}[r] \ar[dr] & TM\oplus T^*M \ar[d]^\rho\\ & TM
 }\]
 sends $[[\ ,\, ]]$ to $[\ ,\,]$.

 Suppose you have $\tilde \pi:T^*M\to TM$, which corresponds to
 $\pi\in \wedge^2 T^{**}M = \wedge^2 TM$.  Given $\pi$, we get an
 almost Dirac structure, which is the graph of $\tilde \pi$.  What
 is the condition on $\pi$ that is equivalent to this being a
 Dirac structure?  That is, when are things like $(\tilde \pi
 \alpha,\alpha)$ closed under the bracket?

 \begin{theorem}
   Given $\pi\in \wedge^2 TM$, define $\{f,g\} = \pi(df,dg)$ on
   $C^\infty M$.  Then the graph of $\tilde \pi$ is a Dirac
   structure if and only if $\{\ ,\, \}$ satisfies the Jacobi
   identity.  In this case, we call $\pi$ a \emph{Poisson
   Structure}.
 \end{theorem}

 If you have a symplectic structure $\tilde\w:TM\to T^*M$, it is
 invertible, so we get a $\tilde \pi$.  Define $\{f,g\} =
 \w(\tilde\w^{-1} df,\tilde\w^{-1}dg)$
