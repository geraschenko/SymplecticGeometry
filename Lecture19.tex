 \stepcounter{lecture}
 \setcounter{lecture}{19}
 \sektion{Lecture 19 - Torus Actions}

 Remark on the paper: the main idea of the paper is to cover a subject, not a
 particular article, or details of proofs.

 There is a book, Toric actions on symplectic manifolds 2nd edition, by Mich\`ele
 Audin, which is very good for the stuff we're doing.

 The subject of torus actions fits into the more general area of completely integrable
 systems.  The basic setup is this.  You have $(M,\w)$ $2n$-dimensional symplectic.
 Following a paper by Duistermaat (1980) in Comm. Pure Appl. Math., we have
 $M^{2n}\xrightarrow{J} W^n$ with rank $n$ almost everywhere, with lagrangian fibres.
 \begin{lemma}
   $J$ has lagrangian fibres if and only if $J$ is a Poisson map, where $M$ has the
   symplectic poisson bracket, and $W$ has the zero poisson bracket.
 \end{lemma}
 That is, $\{f\circ J,g\circ J\}=0$ for all $f,g\in C^\infty(W^n)$.  If $W^n=\RR^n$,
 and $J=(f_1,\dots, f_n)$.  Then this condition says that $\{f_i,f_j\}=0$.  We say
 that these functions are in involution.

 Given $H:M\to \RR$, $J$ is a constant of motion for $X_H$ if and only if $H$ is
 constant on (connected components) of $J$-fibres.  Roughly, $H$ is a function of the
 $f_i$s.  Then we say that this structure forms a completely integrable system.

 There are two steps to understanding these things: understanding $J$ and
 understanding $H$ dynamics.

 Figure 1

 Pick some $\mu\in W$ regular value for $J$, we get a map $T_\mu^*W\to
\chi^1(J^{-1}(\mu))$.  The images of different elements commute.  So the image is a
commuting subalgebra of vector fields spanning $T(J^{-1}(\mu))$.  Take a function on
$W$ and pull it back.  The pull back has a hamiltonian flow along the fibres.  This
map ``integrates'' to an action (perhaps partly defined) of the additive group
$T_\mu^*W$ on $J^{-1}(\mu)$.

Let's assume completeness of the vector fields involved.  e.g. this is implied by the
assumption that $J^{-1}(\mu)$ is compact, which is implied by $J$ proper.  Let's
assume $J$ proper.  Then we get an action of the cotangent space on the fiber (which
is compact), and this action is transitive, locally free, so the fiber $J^{-1}(\mu)$
can be identified with $T^*_\mu W/\Lambda_\mu$, where $\Lambda_\mu$ is a lattice, so
$J^{-1}(\mu)$ must be a torus.

Figure 2

Every fiber of $T^*W$ is a group which acts on the $J^{-1}(\mu)$s, and the isotropy of
all the fibers gives a lattice of 1-forms.  $\bigcup_\mu \Lambda_\mu\subseteq T^*W$ is
locally given by graphs of smooth 1-forms.  Not globally!  The fundamental group
$\pi_1(W,\mu)$ acts on $\Lambda_\mu$.  This action is called ``monodromy''.  Each
fiber in $M$ is a torus.  There is an identification $\Lambda_\mu \simeq
\pi_1(J^{-1}(\mu)) = H_1(J^{-1}(\mu),\ZZ)$.

Note that you have to take out singular values, so you might get some non-simply
connected $W$.  Take the spherical pendulum.  The configuration space is the
two-sphere.  There are two conserved quantities, so we can map $T^*S^2\to \RR^2$ by
$(E,L_z)$, where $E$ is energy, and $L_z$ is angular momentum around the $z$ axis. The
map is proper because the energy levels are compact.  But there are singular values,
and there is monodromy around the singular values.  Cushman, Duistermaat, Vu have
written some stuff.

Now let's go back and analyze integrable systems some more.  If you think a little
bit, any section of $T^*W$ defines a map $M\to M$ by translating the torus fibres by
it.  This is a symplectic map if and only if the section is a closed 1-form.  We
showed that translation by a 1-form is symplectic if and only if it is closed.  The
elements of the lattice act by the identity, so they are smooth closed 1-forms.  I
claim this gives us a rigid structure on $W$.  Let's look on $W$:

Figure 3

\def\x{{\bf{x}}}

We have a lattice of closed 1-forms spanned by $\w^1,\dots, \w^n$ (which are a local
basis).  $\w^j=dx^j$, and since the $\w^i$ are independent, the $x^i$ form a
coordinate system: $\x =(x^1,\dots, x^n):\U\to \RR^n$.  If we choose the $\w^i$, then
the $x^i$ are determined up to a constant.  So we may change $\x\rightsquigarrow
\x+{\bf b}$.  We may also replace $\w$ by $\nu$ by saying that $\nu^j=a^j_i \w^i$,
where $a_i^j\in GL(n,\ZZ)$.  If $\nu^j=dy^j$, then $y^j=a_i^jx^i+b^j$.  We restrict
ourselves to the case where the $\w^i$ form a basis for the lattice.  We get a ``flat
$GL(n,\ZZ)$ structure''.  You can do on $W$ any construction you can do on $\RR^n$ so
long as it is invariant under $GL(n,\ZZ)$.  You can talk about straight lines.  You
can proof that the torsion of the connection is zero.  There is a distinguished family
of bases of the tangent space up to $GL(n,\ZZ)$.  In each tangent space of $W$, there
is a natural integer lattice, and there is a natural identification with nearby
tangent spaces which identifies these lattices.  If the base is compact, it eliminates
certain cases.  If the base is compact, it must be parallelizable, so it cannot be the
2-sphere, for example (in fact, it must be the torus or the klein bottle).  A.T.
Fomenko has written a bunch of stuff about related things.

Locally, $\Lambda \simeq W\times \ZZ^n$, so $M\simeq W\times \TT^n$ locally (in $W$).
The local coordinate function on $W$ obtained from one of these bases generate flows
which rotate the torus.  This is a symplectic isomorphism ... you get coordinates
$(I^1,\dots, I^n, \theta_1,\dots, \theta_n)$, and $\w=dI^j\wedge d\theta_j$.  Then $H$
is a function of the $I^j$s, so hamilton's equations give
\[
    \der{I^j}{t}= 0 \qquad, \qquad \der{\theta_j}{t} = -\underbrace{\pder{H}{I_j}(I_1,\dots, I_n)}_{\phi_j}.
\]
 If $H$ is linear in the $I$s, then it is a very degenerate system ... you get the
 same flow on each torus.

 The non-degenerate case: If $\left(\frac{\partial^2 H}{\partial I_k \partial I_j}\right)$ is
 invertible.  The determinant is $\det(\pder{\phi_j}{I_k})$.  Torus actions are the
 ones for which these lattices are trivial bundles.

 \subsection*{Delzant's Theorem}  $M$ is compact, and $M\xrightarrow{J} {\t^n}^*$ the
 momentum map for an effective torus action.  We're assuming there is no monodromy around the
 singular points.  Then we know that $J(M)=conv(J(M^{\TT}))$.  Note that ${\t^n}^*$
 has a natural $GL(n,\ZZ)$ structure.  The kernel of the exponential map is a natural
 lattice in $\t^n$.  Delzant's theorem says that $J(M)$ is a Delzant polytope, i.e.
 has the following properties:
 \begin{itemize}
   \item[-] From each vertex, there emanate exactly $n$ edges, containing vectors
   which form a basis for the natural lattice $\Lambda$, the dual lattice to the
   kernel of the exponential map.
 \end{itemize}
 \begin{theorem}
   Every Delzant polytope arises as the $J(M)$ for a unique (up to equivariant
   symplectomorphism) hamiltonian torus action on a compact symplectic manifold.
 \end{theorem}
 Notice that the Delzant structure on a polytope doesn't depend on the geometry (you
 can slide the faces of the polytope parallel to themselves).  Consider the
 one-dimensional case.  A delzant polytope is an interval:

 Figure 4

 If we slide the polytope up and down, it doesn't change anything.  If we just move
 one end, it changes the size of the sphere.

 How does the proof go?  To prove that $J(M)$ has these properties is not too bad.
 The hard part is showing that it arises and then showing uniqueness.  The way we do
 that is by exhibiting the manifold $M$ for a polytope $\delta$ as a reduced manifold
 of $\CC^d_\lambda$ for a linear action of $\TT^r$ on $\CC^d$.  How do we get such
 a linear action out of a polytope?  The simplest example is the one-dimensional case.
 The 2-sphere is $\CC\PP^1$, which is $\CC^2_\lambda$ for the Hopf circle action.  In
 general, we have a polytope, so it is the intersection of half-spaces.  $\Delta$ can
 be defined as $\{\mu \in {\t^n}^*|\langle v_j,\mu\rangle \le \lambda_j\}$, where
 $v_1,\dots, v_d\in \t^n$, and $\lambda_j\in \RR$.  The $\{v_j\}$ define a map
 $\RR^d\to \t^n$, which is an integer map (we can choose all the $v_i$ integer).
 $v_j=(v_j^k)$, for $v_j^k\in \ZZ$.  so we get an $d\times n$ integer matrix.  How do
 we build an action of some torus on $\CC^d$ so that the reduced manifold will have
 the right dimension?  We can think of this as a map $\t^n\to \RR^d\simeq \t^d$, so we
 get a $\t^n$ action on $\CC^d$.  We'll do this next time.
