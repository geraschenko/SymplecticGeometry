 \stepcounter{lecture}
 \setcounter{lecture}{8}
 \sektion{Lecture 8 - More}

 It's never too soon to start thinking about the term paper.

 In addition to $T_\CC M$, we have $T_\CC^* M$, the sections of
 which are complex 1-forms.  We will write $dz^j=dx^j+idy^j$.  For
 a (complex) basis for $T_\CC^*(\RR^{2n}$ (at a point), we may
 take either $d_x^j,dy^j$, or $dz^j,d\bar z^j$.

 Note that we have two notions of dual for a complexified vector
 space, but they are naturally isomorphic:
 \[\hom_\CC(V_\CC,\CC)=(V_\CC)^* \cong (V^*)_\CC = \hom_\RR
 (V,\RR)\otimes_\RR \CC.\]

 We can now consider \[\wedge^k T^*_\CC M =
 \wedge^k(\underbrace{T^{*1,0}_M}_{dz^j}\oplus
 \underbrace{T^{*0,1}_M}_{d\bar z^j}) = \bigoplus_{p+q=k} \wedge^p
 T^{*1,0}_M \otimes \wedge^q T^{*0,1}_M\]

 This splitting is naturally attached to the almost complex
 structure.  On the level of sections, we have $\W^k_\CC(M)$.  By
 the way, $(\wedge_\RR^k V)_\CC \simeq \wedge_\CC^k(V_\CC)$, and
 there is a $d_\CC$ taking $k$-forms to $(k+1)$-forms.  In fact,
 the complex de Rham cohomology is the regular de Rham cohomology
 tensored with $\CC$.

 When we have an almost complex manifold, we can write
 \[
    \W^k_\CC(M) = \bigoplus_{p+q=k} \W^{p,q}(M)
 \]
 where $\W^{p,q}(M)$ are forms with $p$ $dz^j$'s and $q$ $d\bar
 z^j$'s (this is cheating, since we don't actually have $z$'s and
 $\bar z$'s).  We can always choose a basis of vector fields
 $(\xi_1,\dots, \xi_n,\eta_1,\dots, \eta_n)$ such that
 $J(\xi_j)=\eta_j$.  Then we get a new basis $\bar\alpha_j =
 \frac{1}{2}(\xi_j+i\eta_j)$ and $\alpha_j =
 \frac{1}{2}(\xi_j-i\eta_j)$.  Then the corresponding basis in the
 dual space are $\bar \theta^j$ and $\theta^j$.  Then a typical
 element $\w\in \W^{p,q}(M)$ is of the form
 \[
    \w = \sum (function) \theta^{i_1}\wedge\cdots\wedge
    \theta^{i_p}\wedge \bar \theta^{i_p}\wedge\cdots\wedge
    \bar\theta^{i_q}
 \]
 Then we have that $d\w$ will, in each term, have either an extra
 $\theta$ or an extra $\bar\theta$.  So $d\w\in
 \W^{p+1,q}(M)\oplus \W^{p,q+1}(M)$, and this decomposition is
 unique, so we may write $d\w = d^{1,0}\w+d^{0,1}\w$.  Thus, we
 have that $d=d^{1,0}+d^{0,1}$.  We often write $d^{1,0}=\delta$
 or $\partial$, and $d^{0,1} = \bar\delta$ or $\bar\partial$.

 \begin{lemma}
   $f$ is [pseudo-]holomorphic if and only if $\bar\partial f=0$.
 \end{lemma}

 We know that $d^2=(\partial+\bar\partial)^2 =
 \partial^2+\partial\bar\partial + \bar\partial\partial
 +\bar\partial^2 =0$.

 \begin{proposition}
   $\partial^2=0$ if and only if $J$ is integrable if and only if
   $\bar\partial^2=0$ if and only if
   $\bar\partial\partial=\partial\bar\partial$.
 \end{proposition}

 In this case, we can just look at the complex
 \[
   0\to \W^{0,0}\xrightarrow{\bar\partial}
    \W^{0,1}\xrightarrow{\bar\partial} \W^{0,2}\xrightarrow{\bar\partial}
 \]
 and the cohomology is called the Dolboex [[sp?]] cohomology.

 Moving from linear algebra to geometry, the statement
 $(V_\CC)^*\simeq (V^*)_\CC$ becomes the following.  Think of
 $\RR^m\subseteq \CC^m$, and consider $C^\w(\RR^m)$, the real
 analytic functions on $\RR^m$, and tensor with $\CC$.  We can
 also consider $C^\w_\CC(\CC^m)$.  If we restrict an analytic
 function to $\RR^m$, you get a complex-valued real analytic
 function.  If you start with a real analytic function on $\RR^m$,
 you can extend it to a neighborhood $U$ of $\RR^m$.
 \[\xymatrix  {
  C^\w(\RR^m)\otimes \CC \ar@/^3mm/[rr]^{\text{\tiny{locally}}} & & \C^\w_\CC(\CC^m) \ar[ll]
 }\]
 figure 1

 If you start with $(\RR^{2n},J)$, with coordinates $x^1,\dots,
 x^{2n}$, you get coordinates
 \begin{align*}
   \eta_j &= \pder{}{x^j} + iJ_j^k(x)\pder{}{x^k}\\
   \xi_j &= \pder{}{x^j} - iJ_j^k(x)\pder{}{x^k}
 \end{align*}
 where $J_j^k(x)$ is the matrix $J:T_x\RR^{2n}\to T_x\RR^{2n}$.
 Then the $\eta_j$ (but only $n$ of them, not
 $2n$\marginpar{hmmmm}) form a basis for the $0,1$ part and the
 $\xi$ for the $1,0$ part.

 $T_{x+i0}\CC^{2n} \simeq T_x\RR^{2n} \oplus iT_x\RR^{2n} \simeq
 T_{\CC,x}\RR^{2n}$.  So we have these complexified vector fields
 on our real manifold.

 If you assume that your almost complex structure is real
 analytic, you can analytically extend all the $J_j^k$ to some
 neighborhood of $\RR^{2n}\subseteq \CC^{2n}$.  So you get some
 sub-bundle $E\subseteq T\CC^{2n}$ (the extension of the $0,1$
 part) of dimension $n$, which is holomorphic (the $\eta_j$ are
 the typical sections of $E$).

 Now let's assume that $J$ is integrable.  This tells is that $E$
 is closed under bracket (it is an \emph{involution}).  Now we
 invoke the holomorphic Frobenius theorem, so we get a foliation
 of some neighborhood of $\RR^{2n}\subseteq\CC^{2n}$.  So near any
 point, we can find $\phi^1,\dots, \phi^n$ independent functions
 which are constant on the leaves of the foliation.  When
 restricted to $\RR^{2n}$, they are still independent (the leaves
 intersection $\RR^{2n}$ transversely) and $J$-holomorphic.

 Since every $0,1$ leaf intersects (at least locally) each $1,0$
 leaf exactly once, you get a map from the leaf to the original
 $\RR^{2n}$.  Since the leaf is complex, this induces a
 holomorphic complex chart on the $\RR^{2n}$.

 In fact, you can start of with something $C^\infty$ and get
 complex coordinates (but this is hard).
